% Welcome to the unofficial template of the master study Smart Systems Engineering at Hanze University of Applied Sciences. Original template created by Natanael Magno Gomes and The Polytechnic Institute of Bragança (IPB).
% Adapted to Hanze's expectations by dr. Felipe Nascimento Martins and subsequently by Ewout Bergsma.
% All information of the "05_Final_report_format_20_21.docx" document is included in this template.
% Last update: 12/12/2021
% Feel free to contact me: ewoutbergsma [at] hotmail.com

\documentclass[12pt,a4paper,twoside]{ipb}
\errorcontextlines=3
\usepackage{csquotes}
\usepackage{amssymb}
\usepackage[portuguese,USenglish]{babel}
\graphicspath{{./images/}}
\usepackage{listings}
\usepackage{xcolor}
\usepackage{realboxes}
\usepackage{url}
\usepackage[colorlinks=true, urlcolor=black, linkcolor=black, citecolor=black, bookmarks=true, pdfstartview=FitH]{hyperref}

\usepackage[style=ieee,backend=biber]{biblatex}
\addbibresource{references.bib}
\usepackage{lipsum}
\usepackage{tabulary}
\usepackage{graphicx}
\usepackage{subfig}
\usepackage{float}  % Positioning of the figures
\usepackage[binary-units=true]{siunitx}
\usepackage[section]{placeins}
\usepackage{indentfirst}  % Indents all paragraphs
\usepackage{mathtools}  % Allows adding multi line equations
\emergencystretch=1em  % Fix overfull
\newcolumntype{P}[1]{>{\centering\arraybackslash}p{#1}}  % Centre col with width
\usepackage{enumitem}
\setlist{nosep}  % By default the separation is large
\renewcommand{\glsgroupskip}{}


\usepackage{fancyref}  % This allows to use \fref{} command. This is very useful, it allows to easily refer to any \label{}. % See documentation: https://mirror.las.iastate.edu/tex-archive/macros/latex/contrib/fancyref/fancyref.pdf.

\usepackage{tabularray}  % This allows to use \tabularray{} command. This is very useful, it allows to easily create tables. % See documentation: https://mirror.las.iastate.edu/tex-archive/macros/latex/contrib/tabularray/tabularray.pdf.
\usepackage{rotating} % This allows to use \rotatebox{} command. This is very useful, it allows to easily create tables. % See documentation: https://mirror.las.iastate.edu/tex-archive/macros/latex/contrib/rotating/rotating.pdf.

\title{Overall Power Optimization of Thread Mesh Wireless Networks}  % Title of your thesis
\author{Md Mazedul Islam Khan}  % Student name
\supervisor{Dr. Mart Johan Deuzeman}
\cosupervisor{Bryan Williams}
\courseyear{2023}
\makeglossaries
\loadglsentries{chapters/acronyms}


\begin{document}
\beforepreface  % Places the cover page

\clearpage
% Navigate to: chapters/abstract.tex
\begin{center}
    ABSTRACT
\vspace{5mm} %5mm vertical space
\end{center}

%%%%%%%%%%%%%%%%%%%%%%%%%%%%%%
This research investigates power optimization in Thread mesh wireless networks, focusing on transmission power as a key parameter. With the increasing prevalence of IoT applications, such as MOOD-Sense, the need for energy-efficient solutions is paramount. We compare the effectiveness of two algorithmic approaches, Monte Carlo (MC) and Genetic Algorithm (GA), in optimizing transmission power to reduce overall power consumption. Our study incorporates two locations, using various network modes and devices.
Results demonstrate that GA consistently outperforms MC in optimizing power consumption, leading to a more energy-efficient network. GA effectively minimizes energy usage by adjusting transmission power based on distance without compromising network performance. In addition to MOOD-Sense, our findings have implications for other IoT applications, promoting more sustainable and energy-efficient implementations.
Our research highlights the importance of responsible and sustainable innovation, emphasizing ethical aspects, reliable services, professional skills, and applied research for system design. The findings contribute to the development of energy-efficient IoT networks, supporting the integrating of devices and systems with minimal environmental impact. This study serves as a foundation for further exploration into power optimization techniques and the expansion of sustainable IoT ecosystems.
%%%%%%%%%%%%%%%%%%%%%%%%%%%%%%

\vspace{5mm} %5mm vertical space
\noindent {\bf Keywords:} Thread mesh network, parameter optimization, power optimization, transmission power, MOOD-Sense.  % Replace keywords


\clearpage
% Navigate to: chapters/declaration.tex
\begin{center}
    DECLARATION
    \vspace{5mm}
\end{center}

I hereby certify that this report constitutes my own product, that where the language of others is set forth, quotation marks so indicate, and that appropriate credit is given where I have used the language, ideas, expressions or writings of another.

\vspace{1em}

I declare that the report describes original work that has not previously been presented for the award of any other degree of any institution.

\vspace{2em}

\noindent\hspace{8cm}Signed,
\begin{figure}[H]
    \hspace{8cm}
    \includegraphics[width=0.15\textwidth]{signature.png}
\end{figure}

\noindent\hspace{8cm}Md Mazedul Islam Khan


\clearpage
% Navigate to: chapters/acknowledgements.tex
\begin{center}
    ACKNOWLEDGEMENTS
    \vspace{5mm} %5mm vertical space
\end{center} 

Here you have the opportunity to thank anyone. Acknowledgements should be in good taste and should not extend more than one page.


\cleardoublepage  % Makes sure contents start on correct page
\afterpreface  % Places contents, list of tables and list of figures
\bodystart  % This sets stylistic parameters for the following content

% Here starts the actual content of your thesis
\chapter{Rationale}\label{chap:rationale}

\section{Introduction}

The research project, titled "Overall Power Optimization of Thread Mesh Wireless Network," is a child project within the broader MOOD-Sense initiative. The MOOD-Sense project employs IoT devices to detect and predict challenging behavior in dementia patients. The project aims to develop an early warning system combining sensors, artificial intelligence, and wireless communication to provide feedback for healthcare professionals and improve patient care and safety \cite{MOOD-Sense_Research}. To further enhance the connectivity and scalability among IoT devices in the MOOD-Sense project, a new network protocol called Thread is proposed to be implemented. Thread is a low-power, IPv6-based, mesh networking protocol specifically designed for IoT applications, offering secure, reliable, and efficient communication. It supports self-healing networks with robust routing capabilities and features like end-to-end encryption, making it an ideal choice for the MOOD-Sense initiative \cite{Thread_Group_Benefits}.

The primary focus of this child project is to optimize the energy efficiency of the wireless Thread network protocol utilized by different wireless sensors and various MOOD-Sense projects. To achieve this, the project examines transmission power network parameter and configuration such as device types, path loss, positions, and RSSI with the objective of determining their optimal configuration.

In order to optimize the transmission power network parameter and reduce overall power consumption, the child project establishes a Thread network and employs an algorithmic approach using appropriate hardware. The ultimate goal is to investigate the impact of transmission power parameter optimization on maintaining reliable communication between devices while minimizing power consumption.


\section{Present Situation}\label{sec:present_situation}
The MOOD-Sense research project originally planned to use three wireless communication technologies: BLE, ZigBee, and Wi-Fi for network communication. However, without a central network protocol, various subprojects within MOOD-Sense, such as dementia patient behavior registration and environmental context monitoring, are being carried out separately. This separation leads to disconnected devices and makes data sharing and integration difficult. The current situation can be visualized as a diagram, showing isolated subprojects and devices without an integrated network.

\begin{figure}[H]
    \centering
    \includegraphics[width=0.8\textwidth]{images/rationale/rationale_present_situation.png}
    \caption{Current state of the MOOD-Sense initiative.}
    \label{fig:rationale_present_situation}
\end{figure}

To address these challenges and create an energy-efficient network, the proposal to implement a Thread mesh wireless network was introduced. Thread's features, such as mesh networking, multiprotocol support, and low cost, make it an ideal solution for connecting BLE, ZigBee, and Wi-Fi connectivity together. By adopting the Thread mesh network protocol, seamless connectivity, interoperability, and communication among all devices within the MOOD-Sense framework can be achieved. This implementation paves the way for the desired outcome of an optimized, low-power, and reliable network.


\section{Desired Outcome}\label{sec:desired_outcome}
The desired outcome of this research is to develop an efficient algorithm that integrates seamlessly within the Thread-based wireless network system and optimizing power consumption. This algorithm will not only adjust transmission power but also select the most suitable device types for the network, contributing to energy efficiency and reliable communication. A schematic overview of the system with the integrated algorithm is as follows:

\begin{enumerate}
    \item \textbf{Input}: The primary input parameters for the network optimization algorithm include the total number of devices and the distance between each device. These inputs provide the necessary data to guide the optimization process and ensure that the algorithm makes informed decisions regarding device types and transmission power levels.
    \item \textbf{Algorithm}: The power optimization algorithm consists of two stages: the Monte Carlo Method and the Genetic Algorithm. In the first stage, MCM focuses on determining the right device types based on various constraints and constructing an optimal network configuration with an initial transmission power setting. The second stage involves GA, which takes the output from MCM and optimizes the transmission power settings to minimize power consumption while maintaining network reliability.
    \item \textbf{Integration}: The algorithm will run separately on a dedicated system and generate output for optimal device types and transmission power settings. This output will then be manually integrated into the Thread devices, ensuring that the devices are configured for optimal performance based on the algorithm's recommendations.
    \item \textbf{Optimization}: In the optimization process, the MCM algorithm initially finds the right device types and builds the optimal network configuration with the initial transmission power, considering various constraints. Afterward, the GA algorithm takes this output and optimizes the transmission power settings to make the network as low-powered and energy-efficient as possible, without compromising on reliability.
    \item \textbf{Output}: The output consists of the appropriate device types for a reliable Thread network configuration, along with the optimal transmission power settings for each device. This output enables the creation of an energy-efficient and reliable Thread-based wireless network that meets the needs of the MOOD-Sense initiative and other similar IoT applications.
\end{enumerate}

By achieving this desired outcome, the algorithm will provide a comprehensive solution for power optimization in Thread networks, supporting the MOOD-Sense initiative and similar IoT applications in building energy-efficient and sustainable networks.


\section{Problem Definition}\label{sec:problem_definition}
As the adoption of IoT devices in applications like the MOOD-Sense initiative increases, there is a growing need for energy-efficient and reliable wireless network protocols. The Thread network protocol offers low-power and reliable mesh networking, making it suitable for such applications. However, optimizing power consumption while maintaining network reliability remains a challenge. Additionally, the selection of appropriate device types is crucial for building an efficient Thread network, as Thread offers various device types depending on the use case.

The primary goal of this research is to determine the most effective algorithmic approach for power optimization in a Thread-based wireless network, specifically through transmission power adjustments and the selection of the right device types. By focusing on these aspects, the research will contribute to the development of energy-efficient network solutions for the MOOD-Sense initiative and similar IoT applications. This approach ensures the proper selection and utilization of devices within the Thread network, optimizing the overall network performance and energy efficiency.


\section{Research Questions}\label{sec:research_questions}

\subsection*{Main Research Question}
How can parameter optimization be applied to develop a power-optimized Thread mesh wireless network?


\subsection*{Sub-Research Questions}
\begin{enumerate}
    \item What are the key features of the Thread protocol that make it suitable for IoT applications, specifically in the context of the MOOD-Sense project?
    \item Which parameters significantly impact the transmission power in a Thread network, and how do they relate to energy efficiency and network performance?
    \item How do the Monte Carlo Method (MCM) and the Genetic Algorithm (GA) differ in their approach to optimizing transmission power in a Thread network, and what are the key steps for their implementation?
    \item What are the specific hardware requirements for implementing a Thread network, and how do they influence the network's energy efficiency and performance in the context of power optimization?
    \item How do variations in algorithmic parameters impact the performance of MCM and GA in optimizing wireless network transmission power and path loss?
    \item Are there any differences in power optimization performance between different iterations for both Maximum and Optimized modes?
    \item How do the different devices perform in terms of power optimization when comparing the Maximum and Optimized modes?
    \item Is there a correlation between mean, max, and min current values and the efficiency of power optimization for both Maximum and Optimized modes?
    \item How does the location impact the power optimization performance of the Maximum and Optimized modes?
    \item How does the performance of the MCM and GA modes differ across different device types, and what impact does this have on current consumption?
    \item How do MCM and GA modes compare with Maximum mode in terms of efficiency across various locations and device types?
    \item What is the significance of errors in the power optimization process and their impact on the performance of MCM and GA modes?
\end{enumerate}


\section{List of Requirements}\label{sec:list_of_requirements}
Focusing on power optimization in the Thread mesh wireless network protocol and ensuring reliable communication, the following requirements address the challenges related to power consumption and sustainability in Thread mesh wireless networks and guide the research process:

\begin{enumerate}
    \item Optimize power efficiency for the Thread network protocol with a focus on minimizing power consumption while maintaining reliable communication.\\
    \textbf{Constraint}: The optimization should not compromise the network's stability or communication quality.
    \item Apply Monte Carlo and Genetic Algorithm for optimizing transmission power and determining efficient network configurations.\\
    \textbf{Constraint}: The optimization techniques should be computationally feasible and should not add significant overhead to the network's operation.
    \item Develop a power-optimized Thread mesh wireless network by considering the optimal device types for different nodes.\\
    \textbf{Constraint}: The selected device types should maintain low power consumption while meeting the network's performance requirements.
    \item Assess the impact of location on power optimization performance for both Maximum and Optimized modes.\\
    \textbf{Constraint}: The assessment should consider diverse environments to ensure the results are applicable to various real-world situations.
    \item Compare the performance of MCM and GA modes across different device types and locations in terms of power optimization.\\
    \textbf{Constraint}: The comparison should be fair and unbiased, taking into account the specific characteristics of each device type and optimization technique.
    \item the significance of errors in the power optimization process and their impact on the performance of MCM and GA modes.\\
    \textbf{Constraint}: The investigation should identify potential sources of errors and recommend ways to minimize their impact on power optimization performance.
    \item Suggest future research directions and improvements for Thread network optimization, including device positioning, path loss, and broader application scope.\\
    \textbf{Constraint}: The suggestions should be realistic and feasible, considering existing limitations and challenges in the field.
    \item Ensure the research adheres to responsible research and innovation principles, including ethical aspects, professional skills, applied research, and sustainability.\\
    \textbf{Constraint}: The research should prioritize the development of sustainable solutions and maintain transparency and accountability throughout the process.
\end{enumerate}

\chapter{Situational \& Theoretical Analysis}\label{chap:situational_theoretical_analysis}
This chapter in your final report can also be the same as in your master thesis definition, but should include all feedback that was given to you.
\chapter{Conceptual Model}\label{chap:conceptual_model}

In this chapter you must explain, using the information from the previous two chapters and critical thinking, the most likely solution to your problem definition (narrowing down the theories and research), resulting in a final theory and/or model. Under a conceptual model it is understood that:  the author first investigates which factors play a role, and then provides argumentation for the identification of the most important ones. This chapter is basically a summary of chapter 2 and should be roughly 1 page long.



\chapter{Research design}\label{chap:research}

The research design section outlines the procedures to accomplish the objectives and enhance the understanding of the subject matter. It presents the methodology as a series of interconnected steps, each crucial for tackling research questions and achieving goals. The following sections further elaborate on these steps and their significance to the research.


\section{Simulate Thread Networks using OTNS and Codelab with Docker}\label{sec:simulation}
The research used OpenThread Network Simulator (OTNS) and Codelab with Docker to simulate Thread networks under different conditions. OTNS is a powerful tool for creating, visualizing, and analyzing virtual Thread networks, while Codelab offers hands-on tutorials for simulating Thread networks in a Docker container environment. This setup allowed accurate emulation of real-world Thread network behavior and efficient execution of numerous simulations. The combination of these tools facilitated a deeper understanding of Thread network principles and mechanisms, enabling the analysis of various configurations and their impact on performance metrics like latency, throughput, and energy consumption, thus paving the way for a more informed optimization process.

\section{Mathematical Constraints for Building a Thread Network}\label{sec:mathematical_constraints}
The objective of the mathematical model is to build a Thread network that adheres to specific mathematical constraints, ensuring a well-functioning network with the required device types, sensitivity, and received signal strength intensity (RSSI). The following mathematical model is designed for this purpose:

\begin{equation}\label{eq:minimize_power}
    \begin{aligned}
        Min\sum_{i=1}^{M}P_t^i
    \end{aligned}
\end{equation}

    \textbf{Subjects to:}
\begin{equation}\label{eq:mathematical_constraints_end_device}
    \begin{split}
        RSSI_{ED}^j>EDSensitivity,j\in1,\cdots,N \\
        -20dbBm{\le P}_t^j\le8dBm \\
        N_{REED}=n_{Router}+n_{Leader} \\
    \end{split}
\end{equation}

\begin{equation}\label{eq:mathematical_constraints_router}
    \begin{split}
        {RSSI}_{Router}^k>RouterSensitivity,k\in1,\cdots,O \\
        -{20dBm\le P}_t^k\le8dBm
    \end{split}
\end{equation}

\begin{equation}\label{eq:mathematical_constraints_border_router}
    \begin{split}
        {RSSI}_{BD}^L>BDSensitivity,L\in1,\cdots,P \\
        -{20dBm\le P}_t^L\le8dBm \\
        Sensitivity=-100dBm \\
        SEED\in0,1 \\
        n_{Leader}=1 \\
        n_{Router}+n_{Leader}\geq3 \\
        n_{BR}=2
    \end{split}
\end{equation}

    Where $P_t$ represents the Transmit Power of each one of the $M$ devices, $RSSI$ is the Received Signal Strength Intensity, $ED$ is End Device, $REED$ is Router Eligible End Devices, $N_{REED}$ is the number of the REEDs, $n_{Router}$ is the number of the routers, $n_{Leader}$ is number of the leaders, $N$ is amount of end devices, $O$ is the number of Routers, $P$ is the number of Border Routers, $SEED$ is Sleepy End Devices, and $Sensitivity$ is $-100 dBm$ with IEEE 802.15.4 \cite{Semiconductor_Nordic_Product_Brief_2018_2.0}.

\vspace{2mm}
The model seeks to minimize the transmitted power of each device while ensuring adequate signal strength, proper device type distribution, and network resilience. By complying with these constraints, the Thread network can be effectively optimized for both performance and energy consumption.

The constraints of the model are as follows:

\vspace{2mm}
\begin{itemize}
    \item To establish a link between the sensors and the End Devices (EDs), the Received Signal Strength Intensity (RSSI) of each ED must be approximately above the sensitivity of each ED. This ensures a stable connection between the devices.
    \item The maximum transmission power of each End Device is limited to $8 dBm$ to minimize energy consumption while maintaining effective communication.
    \item The number of Router Eligible End Devices (REEDs) must be equal to the number of Routers + the Leader because, if a Router is lost, a connected REED must become a Router to replace the "Dead Router" and maintain network resilience.
    \item To establish a connection between the EDs and routers, the RSSI of each Router must be approximately above the Sensitivity of each Router. This guarantees a stable link between the devices and the Routers.
    \item The maximum transmission power of each Router is also limited to $8 dBm$, conserving energy while maintaining efficient communication.
    \item To establish a connection between the Routers and Border Routers, the RSSI of each Border Router must be approximately above the sensitivity of each Border Router, ensuring a reliable link between the network components.
    \item The maximum transmission power of each Border Router is also set to $8 dBm$, optimizing energy consumption while maintaining effective communication within the network.
\end{itemize}
\vspace{3mm}


\section{Monte Carlo Method for Initial Buildup}\label{sec:monte_carlo_method}
Monte Carlo Method is utilized to build the initial Thread network, considering various factors such as the distance between devices, the total number of devices, device types, device positions, and transmission power. The Monte Carlo method comprises several steps, each with its distinct processes.

These steps are separated and explained in detail as follows:

\subsection{Step 1: Start}\label{sec:monte_carlo_method_step_1}
The Monte Carlo method is initiated to optimize the Thread network, considering the key parameters influencing the network's performance and energy efficiency.

These parameters are outlined in the table below:

\begin{longtblr}[
  caption = {Parameters influencing Monte Carlo Method.},
  label = {tab:mcm_parameters},
  ]{
  colspec = {X[5cm] X[10cm]},
  hlines, vlines,
  rowhead = 1, % Repeat the header row on every page
  row{1} = {font=\bfseries},
}
  Parameter & Description \\
  Distance & The distance between the devices \\
  Number of Devices & The total number of devices ($8$ for this research) \\
  Device Types & The type of each device (Unallocated, SEED, REED, Router, Leader, and Border Router) \\
  Device Positions & The position of each device, randomly generated from $1$ to $8$ \\
  Transmission Power & The transmission power of each device, randomly generated between $-20dBm$ to $8dBm$ with $4$ steps difference \\
\end{longtblr}

\subsection{Step 2: Generate Random Numbers}\label{sec:monte_carlo_method_step_2}
Based on the factors mentioned at the start, MCM generates a vector X of length equal to 2n, where n is the number of places where network elements can be allocated. Network elements include all device types and sensors.

The vector is represented as:

\begin{equation}\label{eq:vector_x}
    \begin{split}
        X=\left[x_1,x_2,x_3,\cdots x_n,p_1,p_2,p_3,\cdots,p_n\right] \\
        for{\ x}_n\in0,1,2,3,4,5 \\
        transmission\ power\ p_n\in-20:4:8\ dBm
    \end{split}
\end{equation}

    Where $0$ represents no element allocated, $1$ is allocate a SEED, $2$ is allocate a REED, $3$ is allocate a Router, $4$ is allocate the Leader, and $5$ is allocate a Border Router.
\vspace{2mm}

\subsection{Step 3: Evaluate Results}\label{sec:monte_carlo_method_step_3}
The objective function aims to build a Thread network using the minimized power value of the entire system without violating the constraints specified in the mathematical formulation. If a constraint is violated, a penalty should be added to the objective function, which is weighted according to the importance of the constraint.

The objective function with penalty values can be written as:

\begin{equation}\label{eq:objective_function}
    \begin{aligned}
        Min\sum_{i=1}^{M}P_t^i+penal_1+penal_2+penal_3\ldots+penal_{nr}
    \end{aligned}
\end{equation}

    Where $penal_1$ represents penalty for violating the first restriction, $penal_2$ is penalty for violating the second restriction, and $penal_{nr}$ is the penalty for violating the last restriction.
\vspace{2mm}

A solution is viable only if all constraints are satisfied without penalties. To ensure this, the mathematical model requirements are verified at each step. If a step fails to meet the requirements, the MCM generates new random inputs and re-evaluates the network until a constraint-compliant solution is found. This iterative process guarantees a Thread network that meets performance and energy optimization criteria.

\subsection{Step 4: End}\label{sec:monte_carlo_method_step_4}
The MCM converges on an optimal solution that satisfies necessary constraints, providing outputs such as device types, transmission power, and location. It also offers information on constraint violations, including the penalty, network power consumption, and RSSI sensitivity violations—these outputs aid in understanding the optimization process and refining the network design.


\section{Genetic Algorithm for Transmission Power Optimization}\label{sec:genetic_algorithm}
GA optimizes the Thread network's transmission power to reduce energy consumption while maintaining wireless communication quality. MCM-generated configuration is the starting point for GA.

Key steps and parameters for GA:

\begin{longtblr}[
  caption = {Parameters influencing Genetic Algorithm.},
  label = {tab:ga_parameters},
  ]{
  colspec = {X[5cm] X[10cm]},
  hlines, vlines,
  rowhead = 1, % Repeat the header row on every page
  row{1} = {font=\bfseries},
}
  Parameter & Description \\
  Population Size & The number of individuals in the population \\
  Population & The initial population consisting of transmission power, device types, and device positions \\
  Max Iteration & The maximum number of iterations \\
  Maximum Transmission Power & The minimum transmission power (dBm) \\
  Minimum Transmission Power & The maximum transmission power (dBm) \\
  Mutation Rate & The probability of mutation \\
  Selection Method & The method used for selecting individuals \\
  Mutation Method & The method used for mutating individuals \\
\end{longtblr}

In this table, the "Population" row represents the output from the Monte Carlo simulation, which is a list of transmission power, device types, and device positions.

The GA process includes the following steps:

\vspace{2mm}
\begin{enumerate}
    \item \textbf{}{Initialization:} Generate the initial population using MCM output.
    \item \textbf{}{Selection:} Choose best individuals based on fitness using the "sorted" method.
    \item \textbf{}{Crossover:} Produce offspring by crossing over selected individuals' genes.
    \item \textbf{}{Mutation:} Introduce random changes to transmission power using the "swap" method.
    \item \textbf{}{Evaluation:} Calculate fitness by assessing transmission power and RSSI penalty.
    \item \textbf{}{Termination:} Iterate until reaching the maximum number of iterations.
\end{enumerate}
\vspace{3mm}

The output is a list of optimized transmission power values for each device, along with device types, positions, and penalty values. The solution minimizes transmission power while meeting connectivity constraints.


\section{Prototype Development}\label{sec:prototype_development}
This section details the steps taken to build the prototype that applied the output from Monte Carlo and GA to validate the results. We followed the Monte Carlo output to construct the network by selecting the appropriate device types. Our setup consisted of 8 devices: 2 Border Routers, 3 Routers (with one of them automatically elected as a leader), and 3 Router Eligible Devices (REEDs). The prototype was designed to closely resemble the conceptual model presented earlier in the figure, with the only slight difference being using REEDs instead of sensors as the end devices.

An image is provided below to illustrate the Thread network topology that we have constructed.

\begin{figure}[h]
    \centering
    \includegraphics[width=0.8\textwidth]{images/research_design/prototype_topology.png}
    \caption{Thread network prototye topology.}
    \label{fig:prototype_topology}
\end{figure}

\subsection{Software Implementation}\label{sec:software_implementation}
The nRF-provided Thread Client and Server setup, supporting multiprotocol communication, allowed concurrent communication with Thread and BLE devices. This simplified the process of connecting non-Thread devices to the network. The setup was customized to forward data from Client nodes to the Server using CoAP, validating data transfer within the network. The setup supported both Unicast and Multicast communication out-of-the-box, making it valuable for research purposes.

\subsection{Construction Process}\label{sec:construction_process}
To construct the prototype, we followed these steps:

\vspace{2mm}
\begin{enumerate}
    \item Assigned device roles based on the output from Monte Carlo and GA, ensuring an optimal configuration for the network.
    \item Flashed each router with the Thread Server setup and each REED with the Thread Client setup. In this configuration, routers acted as servers, while REEDs acted as clients. Communication between devices was bidirectional, with the clients having BLE enabled for multiprotocol support.
    \item Flashed the Border Router nodes with the Coprocessor setup provided by nRF. To enable the Raspberry Pi to act as an Edge device, we implemented the OpenThread Radio Coprocessor (RCP) architecture.
    \item Turned on the devices one by one, noting that the first device activated in the network is most likely to become the leader, although leadership can change during the network's lifetime.
    \item Validated all the nodes by running multicast messages using Thread ICMP service. The ICMP service allowed us to send echo requests (ping) to devices, activating their Thread antennas. This allowed us to test the Thread connection, and devices could also reply.
    \item Validated the multiprotocol support connection by running a data flow from the ESP32 UWB devices to the REEDs, which then forwarded the data to the routers. This step ensured seamless communication between non-Thread devices and the Thread network.
    \item Monitored the network for stability and performance, adjusting settings to maintain optimal operation.
\end{enumerate}
\vspace{3mm}

Following these steps, we successfully constructed the prototype to apply the optimized settings obtained from the Monte Carlo simulation and the Genetic Algorithm. The following figure presents a real-world Thread network prototype setup. The image provides a clear view of the nRF52840-based Thread nodes, Raspberry Pi as the Edge device, and the Border Router setup with the Dongle. It also showcases the Development Kits used for Routers and REEDs.

\begin{figure}[h]
    \centering
    \includegraphics[width=0.6\textwidth]{images/research_design/prototype_setup.jpg}
    \caption{Thread network prototype setup in the lab.}
    \label{fig:prototype_setup}
\end{figure}

In the next section, we will discuss the multiprotocol support in more detail, addressing the connection of non-Thread devices with the Thread network.


\subsection{Multiprotocol Support}\label{sec:multiprotocol_support}
The nRF52840 hardware’s multiprotocol support, enabled by the MPSL library, allows integration of non-Thread devices like ESP32 UWB devices. MPSL provides services for multiprotocol apps and facilitates transmission timeslot negotiation. SoftDevice Controller ensures MPSL support as a Bluetooth LE Controller implementation. The dynamic solution enables the simultaneous operation of multiple radio protocols, requiring only radio peripheral reinitialization when switching between protocols \cite{nordic_multiprotocol_support}.

\begin{figure}[h]
    \centering
    \includegraphics[width=0.8\textwidth]{images/research_design/multiprotocol_support.png}
    \caption{Multiprotocol support architecture.}
    \label{fig:multiprotocol_support}
\end{figure}

Figure \ref{fig:multiprotocol_support} shows ESP32 UWB devices connected via BLE to the Thread network, enabled by nRF52840 SoC's multiprotocol support. ESP32 UWB devices bridge non-Thread devices and the Thread network, connecting to REEDs via BLE. This integration highlights network adaptability to various devices. Multiprotocol support expands the device range that can interact with the Thread network, which is crucial in real-world scenarios. Integration of ESP32 UWB devices is possible due to nRF52840 SoC's concurrent support for BLE and Thread. Integrating diverse devices showcases the potential for real-world implementation, particularly in MOOD-Sense applications.


\section{Data Collection}\label{sec:data_collection}
This section will discuss the data collection process for our research, which was conducted in two locations. These locations offered different spaces and distances for the devices, critical factors for our study. We will provide detailed explanations of the locations and distances and an overview of the data collection setup.

\subsection{Locations}\label{sec:locations}
The prototype was set up in two locations: a TechHub Assen lab and a home hallway. These two locations provided different spaces and distances for the devices, which is essential for the research. The lab is small within TechHub Assen, while the home hallway offers a slightly larger space for the devices to operate. By utilizing two different locations, we collected data from diverse environments and better understood our system's performance.

\subsection{Distances}\label{sec:distances}
We considered various distances between devices in our research. The following Euclidean distance matrices represent the distances between devices at each location:

\begin{figure}[h]
    \centering
    \begin{minipage}{0.5\textwidth}
        \centering
        \includegraphics[width=0.9\textwidth]{images/research_design/distance_matrix_lab.png}
        \captionof{figure}{Distance matrix for lab.}
        \label{fig:distance_matrix_lab}
    \end{minipage}%
    \begin{minipage}{0.5\textwidth}
        \centering
        \includegraphics[width=0.9\textwidth]{images/research_design/distance_matrix_home.png}
        \captionof{figure}{Distance matrix for home.}
        \label{fig:distance_matrix_home}
    \end{minipage}
\end{figure}

These distances played a crucial role in our data collection, allowing us to analyze the impact of distance on our system's performance.


\subsection{Data Collection Methodology}\label{sec:data_collection_methodology}
The data collection process utilized various modes, time periods, and data types to analyze system performance, as detailed below:

\vspace{2mm}
\begin{enumerate}
    \item \textbf{Modes: }Two modes were employed: \textit{Maximum} and \textit{Optimized}. Maximum mode operated at $8 dBm$, while Optimized mode used Monte Carlo Method and Genetic Algorithm outputs for transmission power.
    \item \textbf{Time Periods:} 60 $seconds$ was selected for the \textit{Lab} location and 300 $seconds$ for the \textit{Home} location, capturing a larger data set for each location.
    \item \textbf{Data Types:} Data was collected with \textit{No Sensor} connected and with Thread \textit{ICMP Ping}, measuring power consumption under different conditions.
\end{enumerate}
\vspace{3mm}

Data was collected using the nRF Power Profiler Kit II at $100,000$ samples per second. More extended time periods or higher ping rates would have resulted in excessive data, complicating the analysis process.

A table illustrating the data collection process is provided.


% \begin{table}[htbp]
%   \centering
%   \begin{tblr}{
%     colspec={|X[1]|X[1]|X[1]|X[1]|X[1]|X[1]|},
%     hlines, vlines,
%     row{1} = {font=\bfseries},
%   }
%   Location & Mode & Time Period & Data Type & Ping Frequency & Input Voltage \\
%   \SetCell[r=4]{l} Lab & \SetCell[r=2]{l} Maximum & \SetCell[r=4]{l} $60 seconds$ & No Sensor & $0$& \SetCell[r=8]{l} $3.588V$ \\
%   & & & ICMP Ping & $50$ & \\
%   & \SetCell[r=2]{l} Optimized & & No Sensor & $0$ & \\
%   & & & Ping & $50$ & \\
%   \SetCell[r=4]{l} Home & \SetCell[r=2]{l} Maximum & \SetCell[r=4]{l} $300 seconds$ & No Sensor & $0$ & \\
%   & & & ICMP Ping & 290 & \\
%   & \SetCell[r=2]{l} Optimized & & No Sensor & $0$ & \\
%   & & & ICMP Ping & $290$ & \\
%   \end{tblr}
%     \caption{Data collection methodology.}
%     \label{tab:data_collection_methodology}
% \end{table}

\begin{longtblr}[
  caption = {Data collection methodology.},
  label = {tab:data_collection_methodology},
  ]{
  colspec = {|X[1]|X[1]|X[1]|X[1]|X[1]|X[1]|},
  hlines, vlines,
  rowhead = 1, % Repeat the header row on every page
  row{1} = {font=\bfseries},
}
  Location & Mode & Time Period & Data Type & Ping Frequency & Input Voltage \\
  \SetCell[r=4]{l} Lab & \SetCell[r=2]{l} Maximum & \SetCell[r=4]{l} 60 $seconds$ & No Sensor & 0& \SetCell[r=8]{l} 3.588 $V$ \\
  & & & ICMP Ping & 50 & \\
  & \SetCell[r=2]{l} Optimized & & No Sensor & 0 & \\
  & & & Ping & 50 & \\
  \SetCell[r=4]{l} Home & \SetCell[r=2]{l} Maximum & \SetCell[r=4]{l} 300 $seconds$ & No Sensor & 0 & \\
  & & & ICMP Ping & 290 & \\
  & \SetCell[r=2]{l} Optimized & & No Sensor & 0 & \\
  & & & ICMP Ping & 290 & \\
\end{longtblr}

\subsection{Device Setup}\label{sec:device_setup}
This section discusses preparing various device types and modes for data collection.

\subsubsection{Source Meter Mode for Routers and REEDs}\label{sec:source_meter_mode}
For Routers and REEDs, we used the nRF PPK2 in Source Meter mode to measure the current flow and deliver power to the devices. Each device operated at an input voltage of 3.588V. The Source Meter mode allowed us to accurately measure the current flowing through the devices during the data collection.

\begin{figure}[h]
    \centering
    \begin{minipage}[t]{0.45\textwidth}
        \centering
        \includegraphics[width=0.7\linewidth]{images/research_design/PPK2_Router.jpg}
        \captionof{figure}{PPK2 connected to a Router.}
        \label{fig:router_source_meter}
    \end{minipage}\hfill
    \begin{minipage}[t]{0.45\textwidth}
        \centering
        \includegraphics[width=0.7\linewidth]{images/research_design/PPK2_SDK.jpg}
        \captionof{figure}{PPK2 Software in Source Meter mode.}
        \label{fig:ppk2_source_meter}
    \end{minipage}
\end{figure}

Figure \ref{fig:router_source_meter} displays current measurement using the nRF PPK2 from the nRF 52840 DK and Figure \ref{fig:ppk2_source_meter} shows the current measurement in Source Meter mode using the Power Profiler software.

\subsubsection{Ampere Meter Mode for Border Routers}\label{sec:ampere_meter_mode}
Since the USB connection cannot be used with Source Meter mode, and the Border Routers require a connection from the Edge device (Raspberry Pi) to the nRF device to run the RCP (Radio Coprocessor), we used the nRF PPK2 in Ampere Meter mode to measure the current flow for the Border Routers.

\begin{figure}[h]
    \centering
    \begin{minipage}[t]{0.45\textwidth}
        \centering
        \includegraphics[width=0.7\linewidth]{images/research_design/PPK2_Border_Router.jpg}
        \captionof{figure}{PPK2 connected to a Border Router.}
        \label{fig:border_router_ampere_meter}
    \end{minipage}\hfill
    \begin{minipage}[t]{0.45\textwidth}
        \centering
        \includegraphics[width=0.7\linewidth]{images/research_design/PPK2_SDK_Ampere.jpg}
        \captionof{figure}{PPK2 Software in Ampere Meter mode.}
        \label{fig:ppk2_ampere_meter}
    \end{minipage}
\end{figure}

Figure \ref{fig:border_router_ampere_meter} displays current measurement using the nRF PPK2 from the nRF 52840 DK and Figure \ref{fig:ppk2_ampere_meter} shows the current measurement in Ampere Meter mode using the Power Profiler software.

\vspace{2mm}
To accurately measure the current from Border Routers during data collection, the SB40 bridge connection from the SoC was cut so that the current flowed through the PPK2 only and not through the USB. This modification enabled precise current flow measurements.

The prototype, built using specific hardware, was optimized using Monte Carlo simulations and Genetic Algorithms for the transmission power of a Thread network. This provided valuable insights into network performance under various conditions and was a basis for further analysis and validation.

\chapter{Research Results}\label{chap:research_results}

\section{Monte Carlo Method Analysis}

The Monte Carlo Method (MCM) analysis focuses on the output derived from two distinct locations: the Lab and Home. Due to the differences in size between these locations, the distances between devices as input parameter vary, as illustrated in figures \ref{fig:distance_matrix_lab} and \ref{fig:distance_matrix_home}. In the following tables, only the last 5 iterations of the MCM output are presented, as space limitations in this research paper prevent the inclusion of all iterations, which can number in the thousands. For a comprehensive list of the data table, refer to the appendix. The full list of parameters used for the MCM analysis can be found in table \ref{tab:monte_carlo_parameters}.

\begin{longtable}{>{\hspace{0pt}}m{0.313\linewidth}>{\hspace{0pt}}m{0.475\linewidth}>{\hspace{0pt}}m{0.135\linewidth}}
  \label{tab:monte_carlo_results_lab}\\
  \caption{Monte Carlo Method output from lab.}\\
  \hline\hline
  Device                 & $P_{tx} (dBm)$                    & Penalty  \endfirsthead
  \hline
  3, 5, 2, 5, 1, 5, 0, 0 & -20, 0, 0, -8, 0, -12, 0, -20     & 3000     \\
  3, 4, 1, 3, 0, 4, 1, 3 & -16, -8, -4, -20, 0, -12, -4, -20 & 3000     \\
  3, 4, 4, 1, 2, 4, 4, 2 & 4, -12, -20, -4, -8, -12, -20, -8 & 3000     \\
  2, 0, 1, 2, 0, 0, 1, 1 & -12, 4, -8, -8, -20, 0, -8, -20   & 4000     \\
  2, 5, 3, 3, 5, 2, 2, 4 & -8, 8, 0, -16, 0, -8, -20, 8      & 0        \\
  \hline\hline
\end{longtable}

\begin{longtable}{>{\hspace{0pt}}m{0.31\linewidth}>{\hspace{0pt}}m{0.481\linewidth}>{\hspace{0pt}}m{0.133\linewidth}}
  \label{tab:monte_carlo_results_home}\\
  \caption{Monte Carlo Method output from home.}\\
  \hline\hline
  Device                 & $P_{tx} (dBm)$                     & Penalty  \endfirsthead
  \hline
  2, 0, 2, 2, 5, 5, 1, 1 & -12, -8, 8, -4, 0, 8, -20, -12     & 3000     \\
  4, 1, 1, 4, 0, 5, 1, 0 & -12, 4, 0, -8, 8, 8, -12, -20      & 4000     \\
  0, 2, 5, 2, 4, 2, 5, 4 & -20, -4, -12, -12, -4, -8, -20, -4 & 3000     \\
  1, 2, 0, 2, 1, 2, 0, 1 & -8, -8, -16, -16, -12, -8, -16, 4  & 4000     \\
  2, 3, 5, 2, 2, 3, 4, 5 & 8, 8, -20, -8, -4, -16, 4, -16     & 0        \\
  \hline\hline
\end{longtable}

The last row in each table indicates a penalty value of 0, which satisfies the mathematical constraints. When a constraint violation occurs, a penalty value of 1000 is added to the Penalty column. An optimal network configuration, which comprises different device types and initial transmission power, is represented by the absence of a penalty. Table \ref{tab:monte_carlo_results_lab} shows the MCM output from the lab location, where the final row demonstrates an optimal network configuration, with a penalty value of zero. Similarly, table \ref{tab:monte_carlo_results_home} presents the MCM output from the home location, with the last row indicating an optimal configuration, also featuring a penalty value of 0.

Examining the rows with penalties in both tables, we can take the first row as an example. In this row, the penalty value is 3000, indicating three violations. According to the mathematical constraints, a leader router must be present in the network, represented by the number 4 in the device column. The absence of a leader router results in the first violation, adding 1000 to the penalty. The relevant mathematical constraints and models are detailed in equations \ref{eq:minimize_power}, \ref{eq:mathematical_constraints_end_device}, \ref{eq:mathematical_constraints_router}, and \ref{eq:mathematical_constraints_border_router}.

The next constraint requires the number of routers and leaders to be equal to or greater than 3. However, the network configuration in the first row lacks a leader, leading to another violation. Lastly, a constraint mandates that the number of REEDs must be equal to the combined number of routers and leaders. The absence of a leader router in the network configuration causes the penalty value to reach 3000. The Monte Carlo Method continues iterating until it identifies an optimal network configuration without any constraint violations.


\section{Genetic Algorithm Analysis}

Similar to the Monte Carlo Method Analysis, the tables presented below display the output from the Genetic Algorithm for both lab and home locations, with the primary difference between the two scenarios being the distance between devices as input parameters. Unlike the MCM, GA directly provides the final result, showcasing the lowest feasible transmission power without any constraint violations. As GA emphasizes minimizing transmission power, storing all analyzed data from its output is unnecessary, except for the final result. The full list of parameters used in the GA analysis can be found in table \ref{tab:ga_parameters}.

\begin{longtable}{>{\hspace{0pt}}m{0.298\linewidth}>{\hspace{0pt}}m{0.54\linewidth}>{\hspace{0pt}}m{0.087\linewidth}}
  \label{tab:ga_results_lab}\\
  \caption{Genetic Algorithm output from lab.}\\
  \hline\hline
  Device                 & $P_{tx} (dBm)$                           & Total  \endfirsthead
  \hline
  5, 5, 4, 3, 3, 2, 2, 2 & -20, -20, -20, -20, -20, -20, -20, -20 & -160   \\
  \hline\hline
\end{longtable}

\begin{longtable}{>{\hspace{0pt}}m{0.298\linewidth}>{\hspace{0pt}}m{0.54\linewidth}>{\hspace{0pt}}m{0.087\linewidth}}
  \label{tab:ga_results_home}\\
  \caption{Genetic Algorithm output from home.}\\
  \hline\hline
  Device                 & $P_{tx} (dBm)$                         & Total  \endfirsthead
  \hline
  5, 5, 4, 3, 3, 2, 2, 2 & -20, -19, -20, -19, -18, -18, -16, -19 & -149   \\
  \hline\hline
\end{longtable}

The Genetic Algorithm output for the lab location, as shown in table \ref{tab:ga_results_lab}, achieved the lowest possible transmission power of -20 $dBm$ for all nodes in the network. This outcome is expected, given the network's short distances within the small lab setting. When devices are in close proximity to each other, there is no need to increase transmission power, as doing so would waste energy. In this scenario, GA successfully minimized transmission power for all nodes. Although it may appear that GA could have set the power even lower, it's important to note that -20 $dBm$ is the lowest limit, and going below that would depend on the mathematical constraints covering path loss and RSSI sensitivity.

Conversely, table \ref{tab:ga_results_home} displays the GA output for the home location, where transmission power was not set to the lowest possible value for all nodes due to the larger area. The GA output produced a transmission power range from -16 to -20 $dBm$, which is still an impressive result compared to the maximum transmission power mode. This variation in transmission power reflects the differing distances between devices within the home network.

Lastly, the device columns display the same types of devices, which is due to the total number of devices being set to 8, as specified in the parameters table \ref{tab:monte_carlo_parameters}. According to the mathematical constraints, this represents the optimal network configuration derived from the MCM output. If the total number of devices in the network were to be increased, the network configuration would exhibit greater variation.

In addition to these results, examining the plots for both the lab and home locations provides further insight into the transmission power optimization process based on the number of generations. The X-axis of the plots represents the total number of populations, with 100 max populations being set for this research, as mentioned in the parameters table \ref{tab:ga_parameters}. The maximum number of populations is adjusted depending on the optimization process and requirements. The linear curve observed in the plots is influenced by the parameters used in the GA process, such as distances between devices and the selection method. As these factors change, the transmission process is affected, resulting in different curve patterns in the plots. This demonstrates the flexibility of the GA in adapting to various network configurations and optimization objectives.

\begin{figure}[H]
  \centering
  \begin{minipage}[t]{0.5\textwidth}
      \centering
      \includegraphics[width=1\linewidth]{images/research_results/genetic_algorithm_lab_power.png}
      \captionof{figure}{GA transmission power optimization for lab.}
      \label{fig:ga_lab_power}
  \end{minipage}\hfill
  \begin{minipage}[t]{0.5\textwidth}
      \centering
      \includegraphics[width=1\linewidth]{images/research_results/genetic_algorithm_home_power.png}
      \captionof{figure}{GA transmission power optimization for home.}
      \label{fig:ga_home_power}
  \end{minipage}
\end{figure}


\section{Distance vs. Transmission Power Analysis}

Understanding the relationship between distance and transmission power in the network is important for analyzing network configurations. By looking at the plots for both lab and home locations, this relationship becomes clearer. In these plots, the distance between devices is shown on the x-axis, while transmission power is displayed on the y-axis.

\begin{figure}[H]
  \centering
  \begin{minipage}[t]{0.5\textwidth}
      \centering
      \includegraphics[width=1\linewidth]{images/research_results/distance-vs-transmission-power/mcm/lab-distance-vs-txpower.png}
      \captionof{figure}{Transmission power vs. distance for lab location using MCM.}
      \label{fig:lab_distance_vs_txpower_mcm}
  \end{minipage}\hfill
  \begin{minipage}[t]{0.5\textwidth}
      \centering
      \includegraphics[width=1\linewidth]{images/research_results/distance-vs-transmission-power/mcm/home-distance-vs-txpower.png}
      \captionof{figure}{Transmission power vs. distance for home location using MCM.}
      \label{fig:home_distance_vs_txpower_mcm}
  \end{minipage}
\end{figure}

\begin{figure}[H]
  \centering
  \begin{minipage}[t]{0.5\textwidth}
      \centering
      \includegraphics[width=1\linewidth]{images/research_results/distance-vs-transmission-power/ga/lab-distance-vs-txpower.png}
      \captionof{figure}{Transmission power vs. distance for lab location using GA.}
      \label{fig:lab_distance_vs_txpower_ga}
  \end{minipage}\hfill
  \begin{minipage}[t]{0.5\textwidth}
      \centering
      \includegraphics[width=1\linewidth]{images/research_results/distance-vs-transmission-power/ga/home-distance-vs-txpower.png}
      \captionof{figure}{Transmission power vs. distance for home location using GA.}
      \label{fig:home_distance_vs_txpower_ga}
  \end{minipage}
\end{figure}

A closer look reveals that as the distance between devices increases, transmission power also increases. Conversely, devices closer to each other have lower transmission power settings. This pattern is expected because it is inefficient to use extra power for devices that are near each other. Instead, higher transmission power is needed to keep devices connected when they are farther apart.

Transmission power plays a key role in determining the coverage of Thread radio networks. Networks with higher transmission power settings can cover larger areas, ensuring that devices stay connected even when they are separated by greater distances \cite{sheth2002implementation}. Effective power management is important for optimizing network performance and saving energy.


\section{Path Loss Analysis}

The relationship between path loss, distance, and environment is a critical aspect of wireless communication. Two plots are provided to illustrate the path loss between devices at both lab and home locations. As anticipated, these plots demonstrate that the greater the distance between devices, the higher the path loss. It is important to note that even at close distances, higher path loss can occur due to environmental factors, as shown in table \ref{tab:path_loss_exponent}'s path loss exponent.

\begin{figure}[H]
  \centering
  \begin{minipage}[t]{0.5\textwidth}
      \centering
      \includegraphics[width=1\linewidth]{images/research_results/path-loss-lab.png}
      \captionof{figure}{Path loss for lab location.}
      \label{fig:path_loss_lab}
  \end{minipage}\hfill
  \begin{minipage}[t]{0.5\textwidth}
      \centering
      \includegraphics[width=1\linewidth]{images/research_results/path-loss-home.png}
      \captionof{figure}{Path loss for hoome location.}
      \label{fig:path_loss_home}
  \end{minipage}
\end{figure}

Free space environments, such as satellite communication, typically exhibit lower path loss, while indoor locations like homes tend to experience higher path loss due to the presence of obstacles, such as walls and furniture \cite{cho2010mimo}. The plots confirm the expected relationship between path loss, distance, and environment.

The first plot, representing the lab location, displays lower path loss between devices. This can be attributed to the controlled environment and shorter distances between devices. In contrast, the second plot, showcasing the home location, reveals higher path loss between devices. This increase in path loss can be attributed to both the larger distances between devices and the presence of obstacles in the home environment, which impede wireless communication signals.


\section{Algorithmic Parameter Analysis}

So far, the analyses for MCM and GA have been based on the applied values to the experimental prototype, with the only differing parameter being the distance between the two different locations, lab and home. The previous experiments provided valuable insights into the performance of both algorithms within the tested parameters. However, it is also important to explore their behavior with different parameters and larger distances for a more comprehensive understanding. To achieve this, a parameter table has been created using the following imaginary values:

\begin{longtable}{>{\hspace{0pt}}m{0.504\linewidth}>{\hspace{0pt}}m{0.367\linewidth}}
  \label{tab:algorithmic_parameter_analysis}\\
  \caption{Algorithmic parameter analysis.}\\
  \hline\hline
  Parameter        & Value       \endfirsthead
  \hline
  $d$              & 10~$m$       \\
  $D_0$            & 0.35~$m$    \\
  $n$              & 6.0         \\
  $\sigma$         & 5.0~$dB$    \\
  Population size  & 30          \\
  Max iteration    & 20          \\
  Mutation rate    & 0.3         \\
  Selection method & Tournament  \\
  Mutation method  & Random      \\
  \hline\hline
\end{longtable}

These parameters are described in tables \ref{tab:monte_carlo_parameters} and \ref{tab:ga_parameters}. Although increasing the distance would impact the number of devices and the computational power required for the MCM, the chosen distance represents a reasonable compromise for the given number of devices. In response to the sub-research question 5, the tables below present the MCM and GA outputs based on these different parameters, demonstrating how the algorithms behave under different conditions and larger distances:

\begin{longtable}{>{\hspace{0pt}}m{0.327\linewidth}>{\hspace{0pt}}m{0.452\linewidth}>{\hspace{0pt}}m{0.14\linewidth}}
  \label{tab:mcm_different_parameter_analysis}\\
  \caption{Monte Carlo Method output based on different parameters.}\\
  \hline\hline
  Device                 & $P_{tx} (dBm)$                   & Penalty  \endfirsthead
  \hline
  3, 4, 5, 5, 3, 2, 2, 2 & -12, -12, 8, 4, -4, -16, -4, 8 & 0        \\
  \hline\hline
\end{longtable}

\begin{longtable}{>{\hspace{0pt}}m{0.356\linewidth}>{\hspace{0pt}}m{0.452\linewidth}>{\hspace{0pt}}m{0.104\linewidth}}
  \label{tab:ga_different_parameter_analysis}\\
  \caption{Genetic Algorithm output based on different parameters.}\\
  \hline\hline
  Device                 & $P_{tx} (dBm)$               & Total  \endfirsthead
  \hline
  5, 5, 4, 3, 3, 2, 2, 2 & -11, -12, -9, 8, 6, -4, 5, 7 & -10    \\
  \hline\hline
\end{longtable}

While the MCM table may not show any significant differences compared to previous analyses, the transmission power from the GA table does present interesting observations. Upon closer inspection, the transmission power is no longer at the edge range, as seen in past analyses. This outcome is expected since the current transmission power is calculated from much larger distances, while the past analyses were based on closer distances. In smaller distances, lower transmission power values are sufficient to maintain a reliable connection between devices, as the signal propagation is stronger. On the other hand, larger distances require higher transmission power values to ensure the signal can effectively reach and maintain a stable connection with other devices in the network.

In the context of the GA plot for transmission power optimization, the optimization line is no longer linear, likely due to the different parameters used, especially the selection method. The max value reaches a high of 5901 $dBm$ but drops to -17 $dBm$ at the lowest point, a notable result. The Y-axis, which shows the transmission power made up of both penalties for each constraint violation and the output transmission power for each iteration from GA, is understandably higher. Even though the plot reached -17 $dBm$, the best transmission power is the one with no penalty, as shown by the GA output in the table.

\begin{figure}[H]
  \centering
  \begin{minipage}[t]{0.5\textwidth}
      \centering
      \includegraphics[width=1\linewidth]{images/research_results/genetic_algorithm_different_parameter_power.png}
      \captionof{figure}{GA transmission power optimization based on different parameters.}
      \label{fig:genetic_algorithm_different_parameter_power}
  \end{minipage}\hfill
  \begin{minipage}[t]{0.5\textwidth}
      \centering
      \includegraphics[width=1\linewidth]{images/research_results/path-loss-different-parameter.png}
      \captionof{figure}{Path loss based on different parameters for large distance.}
      \label{fig:path_loss_different_parameter}
  \end{minipage}
\end{figure}

In addition to the impact on transmission power, the different parameters also affect path loss. The path loss plot, based on these different network parameters, shows a much higher path loss value as expected. This higher path loss is in line with the trend observed in past analyses, where path loss increased with higher distances. While distance is the primary contributor to these changes, other factors, such as the variance of components $\sigma$, reference distance $d$, and the signal power decay with distance in the path loss model $n$, also play a role. The path loss figure \ref{fig:path_loss_different_parameter} demonstrates these differences, providing valuable insights into how the various parameters influence path loss in different scenarios.


\section{Experimental Data Analysis}

The following analysis delves into a comprehensive data table that compares two distinct modes of operation in communication systems. The primary focus of this data table is to evaluate the current consumption of each mode, with the ultimate goal of identifying the more efficient method for power conservation. To accomplish this, an array of different parameters are considered. The subsequent sections provide an in-depth examination and interpretation of the data, aiming to answer the research questions and offer valuable insights into power optimization strategies.

% https://www.latex-tables.com/

{\tiny
\noindent\begin{minipage}{\linewidth}
  \centering
  \label{tab:experimental_data_analysis}
  \captionof{table}{Experimental data analysis across different scenarios.}\begin{tabular}{l|l|l|l|c|c|r|l|lll|lll}
  \hline\hline
  \multicolumn{1}{l}{\multirow{3}{*}{Method}} & \multicolumn{1}{l}{\multirow{3}{*}{Loc}} & \multicolumn{1}{c}{\multirow{3}{*}{Type}} & \multicolumn{1}{l}{\multirow{3}{*}{Mode}} & \multicolumn{1}{l}{\multirow{3}{*}{Ping}}  & \multicolumn{1}{c}{\multirow{3}{*}{\begin{tabular}[c]{@{}c@{}}$T$\\$(s)$\end{tabular}}} & \multicolumn{1}{c}{\multirow{3}{*}{\begin{tabular}[c]{@{}c@{}}$P_{tx}$\\$(dBm)$\end{tabular}}} & \multicolumn{1}{l}{\multirow{3}{*}{Node}} & \multicolumn{6}{c}{Current consumption~$(mA)$}         \\
  \cline{9-14}
  \multicolumn{1}{l}{}                        & \multicolumn{1}{l}{}                     & \multicolumn{1}{c}{}                      & \multicolumn{1}{l}{}                      & \multicolumn{1}{l}{}                       & \multicolumn{1}{c}{}                                                                    & \multicolumn{1}{c}{}                                                                           & \multicolumn{1}{l}{}                      & \multicolumn{3}{c}{$I_1$} & \multicolumn{3}{c}{$I_2$}  \\
  \cline{9-14}
  \multicolumn{1}{l}{}                        & \multicolumn{1}{l}{}                     & \multicolumn{1}{c}{}                      & \multicolumn{1}{l}{}                      & \multicolumn{1}{l}{}                       & \multicolumn{1}{c}{}                                                                    & \multicolumn{1}{c}{}                                                                           & \multicolumn{1}{l}{}                      & Mean  & Max   & Min       & Mean  & Max   & Min        \\
  \hline
  \multirow{32}{*}{Maximum}                   & \multirow{16}{*}{Lab}                    & \multirow{8}{*}{No Sensor}                & \multirow{32}{*}{N/A}                     & \multirow{8}{*}{0}                         & \multirow{16}{*}{60}                                                                    & \multicolumn{1}{c|}{\multirow{32}{*}{8}}                                                       & BR1                                       & 6.22  & 18.16 & 1.5       & 6.2   & 16.86 & 1.01       \\
                                              &                                          &                                           &                                           &                                            &                                                                                         & \multicolumn{1}{c|}{}                                                                          & BR2                                       & 6.43  & 18.83 & 1.48      & 6.43  & 17.77 & 1.12       \\
                                              &                                          &                                           &                                           &                                            &                                                                                         & \multicolumn{1}{c|}{}                                                                          & R1                                        & 9.72  & 18.83 & 7.3       & 9.78  & 21.22 & 7.44       \\
                                              &                                          &                                           &                                           &                                            &                                                                                         & \multicolumn{1}{c|}{}                                                                          & R2                                        & 9.73  & 19.11 & 7.12      & 9.73  & 20.53 & 7.35       \\
                                              &                                          &                                           &                                           &                                            &                                                                                         & \multicolumn{1}{c|}{}                                                                          & R3                                        & 9.65  & 18.12 & 7.16      & 9.78  & 20.14 & 7.53       \\
                                              &                                          &                                           &                                           &                                            &                                                                                         & \multicolumn{1}{c|}{}                                                                          & ED1                                       & 11.88 & 21.49 & 6.05      & 11.79 & 20.92 & 5.99       \\
                                              &                                          &                                           &                                           &                                            &                                                                                         & \multicolumn{1}{c|}{}                                                                          & ED2                                       & 11.87 & 21.39 & 6.14      & 11.69 & 21.07 & 6.13       \\
                                              &                                          &                                           &                                           &                                            &                                                                                         & \multicolumn{1}{c|}{}                                                                          & ED3                                       & 11.6  & 21.58 & 5.87      & 11.65 & 21.26 & 5.95       \\
  \cline{3-3}\cline{5-5}\cline{8-14}
                                              &                                          & \multirow{8}{*}{Ping}                     &                                           & \multirow{8}{*}{50}                        &                                                                                         & \multicolumn{1}{c|}{}                                                                          & BR1                                       & 6.29  & 17.96 & 1.51      & 6.27  & 18.16 & 1.28       \\
                                              &                                          &                                           &                                           &                                            &                                                                                         & \multicolumn{1}{c|}{}                                                                          & BR2                                       & 6.64  & 20.73 & 1.4       & 6.48  & 19.07 & 1.47       \\
                                              &                                          &                                           &                                           &                                            &                                                                                         & \multicolumn{1}{c|}{}                                                                          & R1                                        & 9.91  & 20.0  & 2.08      & 9.83  & 21.56 & 6.9        \\
                                              &                                          &                                           &                                           &                                            &                                                                                         & \multicolumn{1}{c|}{}                                                                          & R2                                        & 9.82  & 19.27 & 6.62      & 9.75  & 20.87 & 6.76       \\
                                              &                                          &                                           &                                           &                                            &                                                                                         & \multicolumn{1}{c|}{}                                                                          & R3                                        & 9.89  & 19.95 & 6.9       & 9.82  & 20.68 & 6.8        \\
                                              &                                          &                                           &                                           &                                            &                                                                                         & \multicolumn{1}{c|}{}                                                                          & ED1                                       & 11.86 & 21.66 & 6.08      & 11.83 & 22.35 & 6.08       \\
                                              &                                          &                                           &                                           &                                            &                                                                                         & \multicolumn{1}{c|}{}                                                                          & ED2                                       & 11.98 & 22.3  & 6.13      & 11.17 & 21.36 & 6.13       \\
                                              &                                          &                                           &                                           &                                            &                                                                                         & \multicolumn{1}{c|}{}                                                                          & ED3                                       & 11.89 & 21.9  & 6.04      & 11.66 & 21.8  & 5.95       \\
  \cline{2-3}\cline{5-6}\cline{8-14}
                                              & \multirow{16}{*}{Home}                   & \multirow{8}{*}{No Sensor}                &                                           & \multirow{8}{*}{0}                         & \multirow{16}{*}{300}                                                                   & \multicolumn{1}{c|}{}                                                                          & BR1                                       & 6.62  & 18.2  & 1.01      & 6.2   & 17.05 & 1.07       \\
                                              &                                          &                                           &                                           &                                            &                                                                                         & \multicolumn{1}{c|}{}                                                                          & BR2                                       & 6.41  & 18.35 & 1.05      & 6.42  & 17.96 & 1.04       \\
                                              &                                          &                                           &                                           &                                            &                                                                                         & \multicolumn{1}{c|}{}                                                                          & R1                                        & 9.7   & 19.56 & 2.04      & 9.76  & 21.51 & 4.61       \\
                                              &                                          &                                           &                                           &                                            &                                                                                         & \multicolumn{1}{c|}{}                                                                          & R2                                        & 9.65  & 19.12 & 2.28      & 9.72  & 20.82 & 4.56       \\
                                              &                                          &                                           &                                           &                                            &                                                                                         & \multicolumn{1}{c|}{}                                                                          & R3                                        & 9.7   & 19.32 & 4.61      & 9.78  & 20.24 & 2.05       \\
                                              &                                          &                                           &                                           &                                            &                                                                                         & \multicolumn{1}{c|}{}                                                                          & ED1                                       & 11.73 & 21.26 & 5.99      & 11.76 & 21.17 & 6.04       \\
                                              &                                          &                                           &                                           &                                            &                                                                                         & \multicolumn{1}{c|}{}                                                                          & ED2                                       & 11.63 & 21.12 & 5.9       & 11.68 & 20.87 & 5.99       \\
                                              &                                          &                                           &                                           &                                            &                                                                                         & \multicolumn{1}{c|}{}                                                                          & ED3                                       & 11.73 & 21.46 & 5.81      & 11.64 & 21.61 & 5.81       \\
  \cline{3-3}\cline{5-5}\cline{8-14}
                                              &                                          & \multirow{8}{*}{Ping}                     &                                           & \multicolumn{1}{l|}{\multirow{8}{*}{290}}  &                                                                                         & \multicolumn{1}{c|}{}                                                                          & BR1                                       & 6.41  & 19.56 & 0.3       & 6.28  & 18.4  & 1.09       \\
                                              &                                          &                                           &                                           & \multicolumn{1}{l|}{}                      &                                                                                         & \multicolumn{1}{c|}{}                                                                          & BR2                                       & 6.62  & 20.39 & 1.08      & 6.49  & 19.32 & 1.05       \\
                                              &                                          &                                           &                                           & \multicolumn{1}{l|}{}                      &                                                                                         & \multicolumn{1}{c|}{}                                                                          & R1                                        & 9.92  & 20.14 & 4.56      & 9.85  & 21.66 & 2.33       \\
                                              &                                          &                                           &                                           & \multicolumn{1}{l|}{}                      &                                                                                         & \multicolumn{1}{c|}{}                                                                          & R2                                        & 9.83  & 19.61 & 4.65      & 9.76  & 21.07 & 4.65       \\
                                              &                                          &                                           &                                           & \multicolumn{1}{l|}{}                      &                                                                                         & \multicolumn{1}{c|}{}                                                                          & R3                                        & 9.89  & 19.85 & 2.14      & 9.84  & 21.87 & 2.04       \\
                                              &                                          &                                           &                                           & \multicolumn{1}{l|}{}                      &                                                                                         & \multicolumn{1}{c|}{}                                                                          & ED1                                       & 11.9  & 21.85 & 5.99      & 11.84 & 22.64 & 5.99       \\
                                              &                                          &                                           &                                           & \multicolumn{1}{l|}{}                      &                                                                                         & \multicolumn{1}{c|}{}                                                                          & ED2                                       & 11.77 & 21.61 & 5.99      & 11.75 & 21.46 & 5.99       \\
                                              &                                          &                                           &                                           & \multicolumn{1}{l|}{}                      &                                                                                         & \multicolumn{1}{c|}{}                                                                          & ED3                                       & 11.84 & 22.15 & 5.99      & 11.67 & 22.0  & 5.86       \\
  \hline
  \multirow{64}{*}{Optimized}                 & \multirow{32}{*}{Lab}                    & \multirow{16}{*}{No Sensor}               & \multirow{8}{*}{MCM}                      & \multirow{16}{*}{0}                        & \multirow{32}{*}{60}                                                                    & -8                                                                                             & BR1                                       & 6.22  & 16.67 & 1.34      & 6.19  & 16.76 & 1.68       \\
                                              &                                          &                                           &                                           &                                            &                                                                                         & 8                                                                                              & BR2                                       & 6.45  & 18.93 & 1.66      & 6.41  & 18.35 & 1.78       \\
                                              &                                          &                                           &                                           &                                            &                                                                                         & -16                                                                                            & R1                                        & 9.82  & 12.88 & 7.44      & 9.76  & 12.74 & 7.03       \\
                                              &                                          &                                           &                                           &                                            &                                                                                         & 0                                                                                              & R2                                        & 9.71  & 12.83 & 7.39      & 9.7   & 12.83 & 7.35       \\
                                              &                                          &                                           &                                           &                                            &                                                                                         & 0                                                                                              & R3                                        & 9.8   & 12.83 & 7.39      & 9.8   & 20.09 & 7.48       \\
                                              &                                          &                                           &                                           &                                            &                                                                                         & -8                                                                                             & ED1                                       & 11.86 & 15.1  & 6.17      & 11.76 & 15.1  & 6.08       \\
                                              &                                          &                                           &                                           &                                            &                                                                                         & -20                                                                                            & ED2                                       & 11.76 & 14.91 & 6.13      & 11.69 & 14.95 & 6.08       \\
                                              &                                          &                                           &                                           &                                            &                                                                                         & 8                                                                                              & ED3                                       & 11.78 & 21.17 & 6.13      & 11.64 & 21.66 & 5.99       \\
  \cline{4-4}\cline{7-14}
                                              &                                          &                                           & \multirow{8}{*}{GA}                       &                                            &                                                                                         & \multirow{8}{*}{-20}                                                                           & BR1                                       & 6.2   & 16.79 & 1.57      & 6.2   & 16.82 & 1.43       \\
                                              &                                          &                                           &                                           &                                            &                                                                                         &                                                                                                & BR2                                       & 6.42  & 15.88 & 1.6       & 6.41  & 17.17 & 1.57       \\
                                              &                                          &                                           &                                           &                                            &                                                                                         &                                                                                                & R1                                        & 9.77  & 12.88 & 7.3       & 9.76  & 12.83 & 7.39       \\
                                              &                                          &                                           &                                           &                                            &                                                                                         &                                                                                                & R2                                        & 9.68  & 12.83 & 7.12      & 9.73  & 12.83 & 7.44       \\
                                              &                                          &                                           &                                           &                                            &                                                                                         &                                                                                                & R3                                        & 9.76  & 12.6  & 7.12      & 9.78  & 12.79 & 7.21       \\
                                              &                                          &                                           &                                           &                                            &                                                                                         &                                                                                                & ED1                                       & 11.84 & 15.1  & 6.13      & 11.76 & 15.24 & 5.9        \\
                                              &                                          &                                           &                                           &                                            &                                                                                         &                                                                                                & ED2                                       & 11.7  & 14.81 & 6.08      & 11.64 & 15.0  & 6.04       \\
                                              &                                          &                                           &                                           &                                            &                                                                                         &                                                                                                & ED3                                       & 11.73 & 15.0  & 6.04      & 11.7  & 15.05 & 6.04       \\
  \cline{3-5}\cline{7-14}
                                              &                                          & \multirow{16}{*}{Ping}                    & \multirow{8}{*}{MCM}                      & \multirow{16}{*}{50}                       &                                                                                         & -8                                                                                             & BR1                                       & 6.24  & 17.32 & 0.94      & 6.22  & 17.32 & 1.31       \\
                                              &                                          &                                           &                                           &                                            &                                                                                         & 8                                                                                              & BR2                                       & 6.6   & 20.09 & 1.28      & 6.48  & 19.17 & 1.38       \\
                                              &                                          &                                           &                                           &                                            &                                                                                         & -16                                                                                            & R1                                        & 9.78  & 13.4  & 4.78      & 9.8   & 13.07 & 6.9        \\
                                              &                                          &                                           &                                           &                                            &                                                                                         & 0                                                                                              & R2                                        & 9.76  & 13.02 & 2.83      & 9.69  & 13.07 & 6.9        \\
                                              &                                          &                                           &                                           &                                            &                                                                                         & 0                                                                                              & R3                                        & 9.82  & 17.19 & 2.95      & 9.85  & 20.68 & 6.9        \\
                                              &                                          &                                           &                                           &                                            &                                                                                         & -8                                                                                             & ED1                                       & 11.82 & 15.38 & 6.13      & 11.8  & 15.28 & 6.04       \\
                                              &                                          &                                           &                                           &                                            &                                                                                         & -20                                                                                            & ED2                                       & 11.65 & 15.28 & 5.99      & 11.6  & 15.0  & 6.04       \\
                                              &                                          &                                           &                                           &                                            &                                                                                         & 8                                                                                              & ED3                                       & 11.88 & 21.76 & 6.04      & 11.66 & 21.76 & 5.99       \\
  \cline{4-4}\cline{7-14}
                                              &                                          &                                           & \multirow{8}{*}{GA}                       &                                            &                                                                                         & \multirow{8}{*}{-20}                                                                           & BR1                                       & 6.22  & 17.02 & 1.0       & 6.21  & 16.9  & 1.01       \\
                                              &                                          &                                           &                                           &                                            &                                                                                         &                                                                                                & BR2                                       & 6.43  & 17.18 & 1.28      & 6.42  & 16.92 & 1.05       \\
                                              &                                          &                                           &                                           &                                            &                                                                                         &                                                                                                & R1                                        & 9.76  & 13.3  & 4.78      & 9.79  & 13.21 & 7.03       \\
                                              &                                          &                                           &                                           &                                            &                                                                                         &                                                                                                & R2                                        & 9.71  & 13.02 & 4.69      & 9.69  & 13.07 & 6.85       \\
                                              &                                          &                                           &                                           &                                            &                                                                                         &                                                                                                & R3                                        & 9.79  & 13.21 & 5.01      & 9.77  & 13.02 & 6.9        \\
                                              &                                          &                                           &                                           &                                            &                                                                                         &                                                                                                & ED1                                       & 11.81 & 15.28 & 6.04      & 11.77 & 15.28 & 5.99       \\
                                              &                                          &                                           &                                           &                                            &                                                                                         &                                                                                                & ED2                                       & 11.7  & 15.19 & 6.13      & 11.6  & 15.0  & 5.99       \\
                                              &                                          &                                           &                                           &                                            &                                                                                         &                                                                                                & ED3                                       & 11.72 & 15.24 & 5.99      & 11.68 & 15.24 & 5.99       \\
  \cline{2-14}
                                              & \multirow{32}{*}{Home}                   & \multirow{16}{*}{No Sensor}               & \multirow{8}{*}{MCM}                      & \multirow{16}{*}{0}                        & \multirow{32}{*}{300}                                                                   & 8                                                                                              & BR1                                       & 6.22  & 17.58 & 1.04      & 6.2   & 17.29 & 0.98       \\
                                              &                                          &                                           &                                           &                                            &                                                                                         & 8                                                                                              & BR2                                       & 6.42  & 18.16 & 1.06      & 6.41  & 18.4  & 1.12       \\
                                              &                                          &                                           &                                           &                                            &                                                                                         & -8                                                                                             & R1                                        & 9.75  & 13.07 & 2.04      & 9.75  & 12.88 & 4.74       \\
                                              &                                          &                                           &                                           &                                            &                                                                                         & -20                                                                                            & R2                                        & 9.73  & 12.93 & 4.74      & 9.71  & 12.83 & 4.69       \\
                                              &                                          &                                           &                                           &                                            &                                                                                         & -4                                                                                             & R3                                        & 9.75  & 12.83 & 4.65      & 9.79  & 16.3  & 2.95       \\
                                              &                                          &                                           &                                           &                                            &                                                                                         & -16                                                                                            & ED1                                       & 11.75 & 15.1  & 5.95      & 11.75 & 15.1  & 5.99       \\
                                              &                                          &                                           &                                           &                                            &                                                                                         & 4                                                                                              & ED2                                       & 9.73  & 12.93 & 4.74      & 9.71  & 12.83 & 4.69       \\
                                              &                                          &                                           &                                           &                                            &                                                                                         & -16                                                                                            & ED3                                       & 9.75  & 12.83 & 4.65      & 9.79  & 16.3  & 2.95       \\
  \cline{4-4}\cline{7-14}
                                              &                                          &                                           & \multirow{8}{*}{GA}                       &                                            &                                                                                         & -20                                                                                            & BR1                                       & 6.19  & 16.82 & 1.02      & 6.2   & 17.02 & 1.02       \\
                                              &                                          &                                           &                                           &                                            &                                                                                         & -19                                                                                            & BR2                                       & 6.41  & 16.92 & 0.91      & 6.41  & 17.05 & 0.97       \\
                                              &                                          &                                           &                                           &                                            &                                                                                         & -19                                                                                            & R1                                        & 9.76  & 12.83 & 1.99      & 9.75  & 12.93 & 4.78       \\
                                              &                                          &                                           &                                           &                                            &                                                                                         & -20                                                                                            & R2                                        & 9.72  & 12.74 & 2.34      & 9.7   & 12.93 & 2.26       \\
                                              &                                          &                                           &                                           &                                            &                                                                                         & -18                                                                                            & R3                                        & 9.75  & 12.88 & 4.74      & 9.78  & 12.88 & 4.69       \\
                                              &                                          &                                           &                                           &                                            &                                                                                         & -18                                                                                            & ED1                                       & 11.75 & 15.1  & 5.9       & 11.76 & 15.1  & 5.99       \\
                                              &                                          &                                           &                                           &                                            &                                                                                         & -16                                                                                            & ED2                                       & 11.67 & 14.95 & 5.99      & 11.63 & 15.24 & 5.99       \\
                                              &                                          &                                           &                                           &                                            &                                                                                         & -19                                                                                            & ED3                                       & 11.73 & 15.1  & 6.08      & 11.68 & 18.98 & 0.25       \\
  \cline{3-5}\cline{7-14}
                                              &                                          & \multirow{16}{*}{Ping}                    & \multirow{8}{*}{MCM}                      & \multicolumn{1}{l|}{\multirow{16}{*}{290}} &                                                                                         & 8                                                                                              & BR1                                       & 6.42  & 19.17 & 0.33      & 6.28  & 18.25 & 1.04       \\
                                              &                                          &                                           &                                           & \multicolumn{1}{l|}{}                      &                                                                                         & 8                                                                                              & BR2                                       & 6.63  & 19.51 & 1.05      & 6.49  & 19.32 & 1.13       \\
                                              &                                          &                                           &                                           & \multicolumn{1}{l|}{}                      &                                                                                         & -8                                                                                             & R1                                        & 9.73  & 16.45 & 2.04      & 9.8   & 13.21 & 2.29       \\
                                              &                                          &                                           &                                           & \multicolumn{1}{l|}{}                      &                                                                                         & -20                                                                                            & R2                                        & 9.67  & 13.21 & 2.02      & 9.69  & 13.02 & 4.65       \\
                                              &                                          &                                           &                                           & \multicolumn{1}{l|}{}                      &                                                                                         & -4                                                                                             & R3                                        & 9.73  & 16.17 & 2.12      & 9.8   & 13.21 & 4.74       \\
                                              &                                          &                                           &                                           & \multicolumn{1}{l|}{}                      &                                                                                         & -16                                                                                            & ED1                                       & 11.72 & 15.48 & 5.95      & 11.72 & 15.33 & 5.95       \\
                                              &                                          &                                           &                                           & \multicolumn{1}{l|}{}                      &                                                                                         & 4                                                                                              & ED2                                       & 11.69 & 18.11 & 5.9       & 11.65 & 18.25 & 5.95       \\
                                              &                                          &                                           &                                           & \multicolumn{1}{l|}{}                      &                                                                                         & -16                                                                                            & ED3                                       & 11.65 & 15.28 & 5.95      & 11.67 & 15.24 & 5.86       \\
  \cline{4-4}\cline{7-14}
                                              &                                          &                                           & \multirow{8}{*}{GA}                       & \multicolumn{1}{l|}{}                      &                                                                                         & -20                                                                                            & BR1                                       & 6.2   & 17.31 & 1.03      & 6.21  & 16.99 & 1.05       \\
                                              &                                          &                                           &                                           & \multicolumn{1}{l|}{}                      &                                                                                         & -19                                                                                            & BR2                                       & 6.4   & 17.34 & 1.08      & 6.43  & 16.98 & 0.94       \\
                                              &                                          &                                           &                                           & \multicolumn{1}{l|}{}                      &                                                                                         & -19                                                                                            & R1                                        & 9.73  & 13.76 & 2.01      & 9.79  & 13.3  & 2.28       \\
                                              &                                          &                                           &                                           & \multicolumn{1}{l|}{}                      &                                                                                         & -20                                                                                            & R2                                        & 9.68  & 14.25 & 2.03      & 9.69  & 15.24 & 2.21       \\
                                              &                                          &                                           &                                           & \multicolumn{1}{l|}{}                      &                                                                                         & -18                                                                                            & R3                                        & 9.76  & 13.21 & 4.56      & 9.78  & 13.12 & 4.65       \\
                                              &                                          &                                           &                                           & \multicolumn{1}{l|}{}                      &                                                                                         & -18                                                                                            & ED1                                       & 11.74 & 15.28 & 5.99      & 11.78 & 15.33 & 6.08       \\
                                              &                                          &                                           &                                           & \multicolumn{1}{l|}{}                      &                                                                                         & -16                                                                                            & ED2                                       & 11.64 & 15.24 & 5.99      & 11.62 & 15.19 & 5.95       \\
                                              &                                          &                                           &                                           & \multicolumn{1}{l|}{}                      &                                                                                         & -19                                                                                            & ED3                                       & 11.69 & 15.52 & 5.9       & 11.66 & 15.28 & 5.9        \\
  \hline\hline
  \end{tabular}
  \end{minipage}
}

In the analysis of the table \ref{tab:experimental_data_analysis}, a detailed comparison between the maximum and optimized modes can be made, taking into account various parameters, including the mean, max, and min current values, location, iteration, and device type. Here is a more comprehensive overview of the data:


\subsection{Mean, Max, and Min Current Analysis}

The analysis of the mean, max, and min current $(mA)$ across all devices, locations, and methods revealed various trends. Overall, the mean current values ranged from 6.19 $mA$ to 11.98 $mA$, with the lowest values observed in the BR series devices and the highest values in the ED series devices. The maximum current values varied from 12.6 $mA$ to 22.64 $mA$, while the minimum current values were between 0.25 $mA$ and 7.53 $mA$. This broad range of values suggests that different devices, methods, and environments may have significant impacts on the current consumption of the devices tested.


\subsection{Location-Specific Analysis}

The location-specific analysis demonstrated that the devices' performance differed depending on whether they were tested in a lab or at home. In general, the mean, max, and min current values were higher in the lab setting compared to the home setting. This could be attributed to the controlled environment in the lab, which may have led to more stable and consistent performance across devices. This finding answer sub-research question 9, which aims to investigate the impact of location on power optimization performance.


\subsection{Iteration-Specific Analysis}

In comparing the first and second iterations, it was observed that the mean, max, and min current values showed little variation. This indicates that the performance of the devices was consistent across both iterations. However, some minor differences were noticed, such as a slight increase or decrease in the current values for some devices between iterations. This could be due to the variations in the environment or the devices' behavior during the testing period. This analysis answers sub-research question 6, which aims to explore the differences in power optimization performance between different iterations for both maximum and optimized modes.


\subsection{Device-Specific Analysis}

The device-specific analysis revealed that the BR series devices consistently exhibited the lowest mean, max, and min current values compared to the R and ED series devices. In contrast, the ED series devices had the highest mean, max, and min current values. This suggests that the BR series devices may be more energy-efficient than the other devices, while the ED series devices may require more power to operate. This finding answer sub-research question 7, which aims to compare the power optimization performance of different devices in Maximum and Optimized modes.


\subsection{Type-Specific Analysis}

When comparing devices with no sensor versus devices with a ping, it was found that devices without a sensor tended to have slightly lower mean, max, and min current values. This indicates that the presence of a sensor may increase power consumption in certain devices.

Further investigation into this finding revealed that devices with sensors require additional power to operate the sensor and transmit sensor data, leading to increased power consumption. In contrast, devices without sensors do not have these additional power requirements, resulting in lower power consumption overall.


\subsection{Mode-Specific Analysis}

The mode-specific analysis revealed that the devices' performance was affected by the MCM and GA modes. In general, the mean, max, and min current values were higher in the MCM mode compared to the GA mode. This suggests that the MCM mode require more power to operate and maintain, while the GA mode may offer more energy-efficient performance.

Furthermore, when comparing the power optimization performance of the MCM and GA modes, it was found that the GA mode outperformed the MCM mode in terms of energy efficiency. Devices operating in the GA mode consumed less power while still achieving comparable levels of performance, indicating that this mode may be a better option for power optimization. This finding supports the sub-research question 10 on the effect of mode on power optimization, indicating that the choice of mode can have a significant impact on power consumption and optimization performance.


\subsection{Method-Specific Analysis}

Lastly, the method-specific analysis showed that the mean, max, and min current values were lower in the optimized method compared to the maximum method. This indicates that the optimized method may provide a more energy-efficient solution for the devices tested, as it consumes less power overall. This insight could prove useful when selecting a method for future deployments to reduce energy consumption and improve device performance. This finding answer sub-research question 8, which aims to explore the correlation between mean, max, and min current values and the efficiency of power optimization for different methods.

\vspace{3mm}
In conclusion, the data analysis of the mean, max, and min current values across various parameters, including iteration, device type, location, sensor presence, mode, and method, revealed distinct trends in the devices' power consumption. The BR series devices consistently showed the lowest current values, suggesting better energy efficiency compared to other device types. Devices tested in the lab displayed higher current values than those in the home setting, indicating the influence of environmental factors on device power consumption.

Furthermore, devices without sensors generally consumed less power, and the GA mode demonstrated lower current values compared to the MCM mode. Finally, the optimized method appeared to be a more energy-efficient solution compared to the maximum method.


\section{Experimental Results}

In this section, the power efficiency of the devices under study is examined, focusing on the maximum and optimized methods applied to MCM and GA modes. The analysis considers the error values obtained from the first and second iteration MCM and GA values, providing insights into the variability and precision of the measurements. This results-driven perspective allows for a comprehensive understanding of the performance differences between the methods and modes under investigation.

\begin{longtable}{lllllllll}
  \label{tab:experimental_results}\\
  \caption{Experimental results across different scenarios with errors.}\\
  \hline\hline
  \multirow{4}{*}{Method}    & \multirow{4}{*}{Location} & \multirow{4}{*}{Type} & \multicolumn{4}{c}{Iteration~$(\%)$}                  & \multicolumn{2}{l}{\multirow{3}{*}{Error~$(\%)$}} \\*
  \cline{4-7}
                             &                           &                       & \multicolumn{2}{c}{$I_1$} & \multicolumn{2}{c}{$I_2$} & \multicolumn{2}{l}{}                              \\*
  \cline{4-7}
                             &                           &                       & \multicolumn{4}{c}{Mode}                              & \multicolumn{2}{l}{}                              \\*
  \cline{4-9}
                             &                           &                       & MCM   & GA                & MCM   & GA                & MCM  & GA                                         \\*
  \hline
  \multirow{4}{*}{Optimized} & \multirow{2}{*}{Lab}      & No Sensor             & 25.69 & 26.42             & 20.6  & 26.31             & 5.09 & 0.11                                       \\*
                             &                           & Ping                  & 18.52 & 27.07             & 18.39 & 28.47             & 0.13 & 1.4                                        \\*
  \cline{2-9}
                             & \multirow{2}{*}{Home}     & No Sensor             & 27.12 & 25.92             & 24.38 & 24.25             & 2.74 & 1.67                                       \\*
                             &                           & Ping                  & 19.24 & 26.19             & 24.84 & 27.47             & 5.6  & 1.28                                       \\
  \hline\hline
\end{longtable}

The table \ref{tab:experimental_results} presents a comprehensive comparison of MCM and GA modes in the Optimized method, with a focus on the percentage values calculated from the maximum current values obtained in previous analyses. Considering the maximum current values for the power optimization process is important because it helps identify the devices' peak current usage. This method offers a clearer understanding of the devices' power efficiency in different modes and iterations. By focusing on the highest current values, a more accurate assessment of the effectiveness of the power optimization process can be achieved, especially during the most demanding situations. This approach addresses the sub-research question 11, which aims to understand the impact of power optimization methods on devices' power efficiency across different modes and iterations.

In the optimized method, the lab location exhibits a higher percentage for no sensor and ping types in both MCM and GA modes when compared to the home location. Specifically, for the no sensor type, the lab location has a 25.69\% and 26.42\% improvement in MCM and GA modes, respectively, in the first iteration, while for the ping type, the lab location has an 18.52\% and 27.07\% improvement in MCM and GA modes, respectively. In the second iteration, the lab location maintains its higher performance with 20.6\% and 26.31\% improvements in MCM and GA modes for the no sensor type, and 18.39\% and 28.47\% improvements in MCM and GA modes for the ping type. This observation indicates that the devices in the lab location demonstrate better power efficiency.

On the other hand, in the home location, the no sensor type shows a 27.12\% and 25.92\% improvement in MCM and GA modes, respectively, in the first iteration, while for the ping type, there is a 19.24\% and 26.19\% improvement in MCM and GA modes, respectively. In the second iteration, the home location has a 24.38\% and 24.25\% improvement in MCM and GA modes for the no sensor type, and 24.84\% and 27.47\% improvements in MCM and GA modes for the ping type.

Errors were calculated based on the differences between the first and second iteration values for MCM and GA modes. The presence of errors might be attributed to various factors, such as device inconsistencies, environmental factors, or potential limitations in the experimental setup. These errors affect the research by introducing a level of uncertainty in the results, making it necessary to interpret the findings with caution, which addresses the second sub-research question 12. For instance, in the lab location, the no sensor type has errors of 5.09\% and 0.11\% in MCM and GA modes, while in the home location, errors are 2.74\% and 1.67\% for the same modes.

In conclusion, the analysis demonstrates the effectiveness of parameter optimization in developing a power-optimized Thread mesh wireless network, addressing the main research question and the problem definition. Both MCM and GA modes outperform the maximum method, with GA optimization consistently offering better optimization results than MCM across different locations and device types. This indicates that the GA approach significantly contributes to lowering power consumption in Thread mesh wireless networks by optimizing transmission power parameter more effectively than the MCM method.

The algorithmic approach, specifically the GA optimization, can be integrated into the system by adjusting transmission power parameter according to the optimization results. By monitoring the network conditions and transmission power, the Thread mesh wireless network can maintain optimal energy efficiency. The results provide a solid foundation for future exploration and enhancements in power optimization using algorithmic approaches, addressing the challenges of consuming higher power, ultimately realizing the full potential of Thread-based wireless communication in a wide range of low-powered IoT network fields.

\chapter{Conclusions and Recommendations}\label{cap:conclusions_recommendations}

\section{Conclusions}\label{sec:conclusions}
This research on power optimization in Thread mesh wireless networks using transmission power as a parameter has demonstrated the effectiveness of algorithmic approaches, particularly Genetic Algorithm, in reducing power consumption. Genetic Algorithm optimization consistently outperformed both Monte Carlo Method mode and maximum method across different locations and device types, with improvements of up to 28.47\% in power efficiency and error rates as low as 0.11\%. Monte Carlo Method also achieved improvements of up to 27.12\% in power efficiency, while errors reached up to 5.6\%. These results not only enhance the performance of MOOD-Sense initiatives and other IoT applications but also contribute to sustainable and energy-efficient IoT network implementation. By adhering to responsible research and innovation principles, this study ensures the development of an optimized system design adaptable for various applications beyond MOOD-Sense, promoting energy-conserving, environmentally friendly, and sustainable IoT devices and network integration. This research demonstrates that optimizing transmission power using algorithmic approaches, specifically Genetic Algorithm optimization, can significantly reduce power consumption in Thread mesh wireless networks, paving the way for future exploration and enhancements in power optimization using algorithmic approaches, addressing the challenges of consuming higher power, and ultimately realizing the full potential of Thread-based wireless communication in a wide range of low-powered fields.


\section{Recommendations}\label{sec:recommendations}
Considering the conclusions from this research, several recommendations for future work are proposed to further enhance power optimization in Thread mesh wireless networks. These suggestions aim to build on the foundation laid by this research and contribute to the ongoing development of Thread mesh wireless networking technologies.

\begin{enumerate}
    \item \textbf{Dynamic Transmission Power Allocation}: Develop a custom SDK on top of existing platforms like Zephyr, nRF, or OpenThread that automatically sets the transmission power based on the distances between devices without requiring manual action and reflashing the device. By automating this process, the network can achieve better energy efficiency, adapt to changes in device locations more effectively, and minimize the need for human intervention to update transmission power settings, making the Thread network more sustainable and user-friendly.
    \item \textbf{Exploring Different Thread Devices}: Investigate the impact of different Thread devices, such as Full Thread Devices (FTD), Minimal Thread Devices (MTD), and Sleepy Thread Devices (STD), on power consumption. By understanding the unique characteristics and energy requirements of each device type, the most suitable Thread devices can be selected to improve overall network efficiency. A thorough evaluation of device capabilities, power requirements, and application-specific needs can help guide the selection process for an optimized network configuration.
    \item \textbf{Investigating Low-Power SoC Options}: Assess various low-power System-on-Chip (SoC) options available on the market to determine the most energy-efficient solutions for the Thread network. By considering different devices with better low-powered SoC capabilities, the overall energy consumption of the network can be reduced, leading to a more sustainable and efficient network. This exploration can help identify devices that meet the performance requirements of the network while minimizing power consumption and maximizing energy efficiency.
\end{enumerate}

Implementing these recommendations can help future research advance the optimization of Thread mesh wireless networks, ultimately leading to more efficient IoT wireless networking solutions.

\chapter*{Definitions and Abbreviations}\label{chap:additional_chapters}

\addcontentsline{toc}{chapter}{Definitions and Abbreviations}

Here is a list of abbreviations used throughout the research, along with their short definitions:

\begin{enumerate}
    \item \textbf{IoT} - Internet of Things: A network of interconnected devices and sensors that communicate and share data over the internet.
    \item \textbf{MCU} - Microcontroller Unit: A small, integrated computer on a single chip, used for embedded systems and control applications.
    \item \textbf{UWB} - Ultra-Wideband: A radio technology that uses a wide frequency range for high data rate, short-range communication.
    \item \textbf{MCM} - Monte Carlo Method: A computational method that uses random sampling to solve complex problems and estimate results.
    \item \textbf{GA} - Genetic Algorithm: A search heuristic inspired by natural selection and genetics, used to optimize complex problems.
    \item \textbf{mA} - milliampere: A unit of electric current equal to one-thousandth of an ampere.
    \item \textbf{uA} - microampere: A unit of electric current equal to one-millionth of an ampere.
    \item \textbf{dBm} - decibel-milliwatts: A unit of power level used to express the ratio of power in decibels relative to 1 milliwatt.
    \item \textbf{WSN} - Wireless Sensor Network: A network of small, low-power devices that communicate wirelessly.
    \item \textbf{RF} - Radio Frequency: Electromagnetic wave frequencies used for wireless communication.
    \item \textbf{dB} - decibel: A unit of measurement used to express the ratio of two values in a logarithmic scale.
    \item \textbf{FTD} - Full Thread Device: A node in a Thread network capable of routing traffic and participating in network management functions.
    \item \textbf{MTD} - Minimal Thread Device: A node in a Thread network with limited routing capabilities, primarily functioning as an end device for low-power operation.
    \item \textbf{STD} - Sleepy Thread Device: A low-power node in a Thread network that periodically wakes up to communicate before returning to sleep mode.
    \item \textbf{SoC} - System on a Chip: An integrated circuit that combines multiple electronic components, such as a microprocessor, memory, and interfaces, on a single chip.
    \item \textbf{nRF} - Nordic Semiconductor's RF series: A family of wireless SoCs developed by Nordic Semiconductor, designed for ultra-low power consumption and high-performance wireless connectivity.
\end{enumerate}


\printglossary[type=\acronymtype,title={Definitions and Abbreviations}]

\clearpage
\addcontentsline{toc}{chapter}{References}
\printbibliography[title={References}]

\chapter*{Appendix}\label{appendix}
% Following is required as above uses \chapter*{} (note the star). The start makes the chapter unnumbered, but also removes it from table of content. Former is desired, the latter is not:
\addcontentsline{toc}{chapter}{Appendix}


% Temporarily remove the chapter number from the section counter
\counterwithout{section}{chapter}


\appendix
This Appendix provides various resources and links related to the research project. These resources include the full dataset analysis, algorithms, custom implementations, and output datasets. Due to these materials' large size and complexity, it is not feasible to include them directly in the research paper. Instead, the links in the following sections grant access to the complete datasets, algorithms, and implementations, allowing interested readers to explore the project in greater detail and better understand the methodology, optimization techniques, and findings. The sections below outline the resources available in the appendix.

\section{Dataset Analysis}
This section provides the link to the dataset analysis repository on GitLab. This repository contains a comprehensive set of analyses performed for the project. Due to the extensive nature of the analyses, including them all in this paper is not feasible. By sharing the repository, readers can access detailed studies and better understand the project's intricacies. The repository can be accessed using the following link: \url{https://gitlab.com/mmikhan/threadpowerprofiler/}

\section{Dataset}
The complete dataset, too large to include within the research paper, is available on GitLab. This dataset contains detailed information on the performance of the Thread network under various conditions and configurations. The original dataset is in binary format but has been converted to CSV for convenience and easier access. Access the dataset here: \url{https://gitlab.com/mmikhan/threadpowerprofiler/}

\section{Algorithm}
This section links the complete algorithm consisting of the \gls{MCM} and \gls{GA} implementations on GitHub. This repository houses the code and documentation required to understand and replicate the optimization techniques used in this research project. Access the algorithm here: \url{https://github.com/mmikhan/ThreadNetPowerOptGA}

\section{nRF Thread Client and Server Custom Implementation}
This part presents the custom implementation of the \gls{nRF} Thread Client and Server used in the physical prototype. This implementation was essential to successfully deploying and testing the optimized Thread network. Access the \gls{nRF} Thread Client and Server custom implementation here: \url{https://github.com/mmikhan/Connecta}

\section{Optimization Results}
Finally, this section provides access to the large output dataset from \gls{MCM} and \gls{GA} simulations. This dataset is crucial for understanding the outcomes of the optimization techniques and their impact on the energy efficiency and performance of the Thread network. Access the output dataset here: \url{https://gitlab.com/mmikhan/threadpowerprofiler/}

\vspace{2mm}
The resources presented in the appendix thoroughly examine the research project, its methodology, and the optimization techniques utilized. Through carefully studying these materials, a comprehensive understanding of the project's development, implementation, and outcomes can be obtained, thereby enriching the overall context of the research.


% Restore the chapter number to the section counter
\counterwithin{section}{chapter}
 % Note, appendix must be last

\end{document}
