\begin{center}
    ABSTRACT
\vspace{5mm} %5mm vertical space
\end{center}

%%%%%%%%%%%%%%%%%%%%%%%%%%%%%%
This research investigates power optimization in Thread mesh wireless networks, focusing on transmission power as a key parameter. With the increasing prevalence of IoT applications, such as MOOD-Sense, the need for energy-efficient solutions is paramount. We compare the effectiveness of two algorithmic approaches, Monte Carlo (MC) and Genetic Algorithm (GA), in optimizing transmission power to reduce overall power consumption. Our study incorporates two locations, using various network modes and devices.
Results demonstrate that GA consistently outperforms MC in optimizing power consumption, leading to a more energy-efficient network. GA effectively minimizes energy usage by adjusting transmission power based on distance without compromising network performance. In addition to MOOD-Sense, our findings have implications for other IoT applications, promoting more sustainable and energy-efficient implementations.
Our research highlights the importance of responsible and sustainable innovation, emphasizing ethical aspects, reliable services, professional skills, and applied research for system design. The findings contribute to the development of energy-efficient IoT networks, supporting the integrating of devices and systems with minimal environmental impact. This study serves as a foundation for further exploration into power optimization techniques and the expansion of sustainable IoT ecosystems.
%%%%%%%%%%%%%%%%%%%%%%%%%%%%%%

\vspace{5mm} %5mm vertical space
\noindent {\bf Keywords:} Thread mesh network, parameter optimization, power optimization, transmission power, MOOD-Sense.  % Replace keywords
