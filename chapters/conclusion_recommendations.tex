\chapter{Conclusions and Recommendations}\label{cap:conclusions_recommendations}

\section{Conclusions}\label{sec:conclusions}
This research on power optimization in Thread mesh wireless networks using transmission power as a parameter has demonstrated the effectiveness of algorithmic approaches, particularly Genetic Algorithm, in reducing power consumption. Genetic Algorithm optimization consistently outperformed both Monte Carlo Method mode and maximum method across different locations and device types, with improvements of up to 28.47\% in power efficiency and error rates as low as 0.11\%. Monte Carlo Method also achieved improvements of up to 27.12\% in power efficiency, while errors reached up to 5.6\%. These results not only enhance the performance of MOOD-Sense initiatives and other IoT applications but also contribute to sustainable and energy-efficient IoT network implementation. By adhering to responsible research and innovation principles, this study ensures the development of an optimized system design adaptable for various applications beyond MOOD-Sense, promoting energy-conserving, environmentally friendly, and sustainable IoT devices and network integration. This research demonstrates that optimizing transmission power using algorithmic approaches, specifically Genetic Algorithm optimization, can significantly reduce power consumption in Thread mesh wireless networks, paving the way for future exploration and enhancements in power optimization using algorithmic approaches, addressing the challenges of consuming higher power, and ultimately realizing the full potential of Thread-based wireless communication in a wide range of low-powered fields.


\section{Recommendations}\label{sec:recommendations}
Considering the conclusions from this research, several recommendations for future work are proposed to further enhance power optimization in Thread mesh wireless networks. These suggestions aim to build on the foundation laid by this research and contribute to the ongoing development of Thread mesh wireless networking technologies.

\begin{enumerate}
    \item \textbf{Dynamic Transmission Power Allocation}: Develop a custom SDK on top of existing platforms like Zephyr, nRF, or OpenThread that automatically sets the transmission power based on the distances between devices without requiring manual action and reflashing the device. By automating this process, the network can achieve better energy efficiency, adapt to changes in device locations more effectively, and minimize the need for human intervention to update transmission power settings, making the Thread network more sustainable and user-friendly.
    \item \textbf{Exploring Different Thread Devices}: Investigate the impact of different Thread devices, such as Full Thread Devices (FTD), Minimal Thread Devices (MTD), and Sleepy Thread Devices (STD), on power consumption. By understanding the unique characteristics and energy requirements of each device type, the most suitable Thread devices can be selected to improve overall network efficiency. A thorough evaluation of device capabilities, power requirements, and application-specific needs can help guide the selection process for an optimized network configuration.
    \item \textbf{Investigating Low-Power SoC Options}: Assess various low-power System-on-Chip (SoC) options available on the market to determine the most energy-efficient solutions for the Thread network. By considering different devices with better low-powered SoC capabilities, the overall energy consumption of the network can be reduced, leading to a more sustainable and efficient network. This exploration can help identify devices that meet the performance requirements of the network while minimizing power consumption and maximizing energy efficiency.
\end{enumerate}

Implementing these recommendations can help future research advance the optimization of Thread mesh wireless networks, ultimately leading to more efficient IoT wireless networking solutions.
