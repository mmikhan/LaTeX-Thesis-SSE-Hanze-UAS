\chapter{Research design}\label{chap:research}

The Research Design chapter contains everything that you have done in order to answer your research question. Write and state things in such a way that somebody can repeat your research and arrive at the exact same results. You can use the chapter in your master thesis definition as a start, but generally, research designs end up differently from what you intended to do in the first place. A special remark: it should be written in the PAST TENSE, since you now have already done it.


Note, the following content are some examples of techniques that one can use. Read the comments for explanations.


\section{Methodology}

A list of requirements has been defined to guide the project, a general and detailed design were developed and implemented to support the investigation of the research question.

\subsection{Requirements}

% One can replace \begin{enumerate} and \end{enumerate} with \begin{itemize} and \end{itemize} respectively for bullets instead of numbers.
\vspace{2mm}
\begin{enumerate}
  \item The Cobot safety features are enabled at all times.
  \item The system architecture is ROS based.
  \item The system works in simulation and real environment.
  \item The training is done using RL techniques.
  \item The algorithm is trained using self experience and memory replay.
  \item The algorithm uses RGBD images to estimate the actions.
\end{enumerate}
\vspace{3mm} 

\section{General Design}

General design text. See \fref{tab:models}.  % Note how \fref{} allows to refer to a label.

\section{Detailed Design}

Detailed design text. You don't have to follow this structure. Feel free to include, remove or change the titles of the sections.

An example of a table is shown in \fref{tab:models}. Moreover, an example of a figure is shown in \fref{fig:example_figure}. Always use the proper reference to cite tables and figures. Do not mention "table below" or "figure above", for example.

\begin{table}
    \centering
        \begin{tabular}{ p{35mm}  P{3cm}   P{40mm}  P{3cm}}
            \hline
            Model name & Feature space & Est. total size (MB) & Top-5 error \\
            \hline
            AlexNet          &   256  x 6 x  6     &  242.03  &    20.91 \\
            SqueezeNet       &   512  x 13 x 13   &   76.54   &    19.58 \\
            Densenet         &   1024 x 7 x  7    &   325.21  &    7.83  \\
            GoogleNet        &   1024 x 7 x  7    &   85.91   &    10.47 \\
            MobileNet        &   1280 x 7 x  7    &   162.45  &    9.71  \\
            ResNeXt-50-32x4d &   2048 x 7 x  7    &   415.72  &    6.30  \\
            MNASNet          &   1280 x 7 x  7    &   135.77  &    8.4  \\
            \hline
        \end{tabular}
    \caption{This is the caption of the example table \cite{torchvisionmodels}.}
    \label{tab:models}
\end{table}


\begin{figure}[b]  % b for bottom of page
    \centering
    \includegraphics[width=0.8\textwidth]{images/hanzelogo_nl.png}
    \caption{This is the caption of the example figure.}
    \label{fig:example_figure}
\end{figure}


