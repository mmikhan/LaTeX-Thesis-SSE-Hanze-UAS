\chapter{Research Design}\label{chap:research_design}

\section{Mathematical Constraints}

The objective of the mathematical model is to build a Thread network that adheres to specific mathematical constraints, ensuring a well-functioning network with the optimum device types, sensitivity, and \gls{RSSI}. By complying with the constraints, the Thread network can be effectively optimized for both performance and energy consumption. The constraints of the model are as follows:

\begin{enumerate}
    \item To establish a link between the devices within the network, the \gls{RSSI} of each device must be approximately above the sensitivity of the device, which is -100 $dBm$ with \gls{IEEE} 802.15.4 \cite{Semiconductor_Nordic_Product_Brief_2018_2.0}. This ensures a stable connection between the devices \cite{wu2014study}.
    \item The transmission power limitation for all types of devices, ranging from -20 $dBm$ to 8 $dBm$, is set according to the hardware specifications of the devices in the Thread network, ensuring optimal performance while facilitating power optimization techniques within these constraints for energy efficiency \cite{semiconductor_nrf52840_2018_1}.
    \item The number of \glspl{REED} must be equal to the number of routers and the leader because, if a router is lost, a connected \gls{REED} must become a router to replace the dead router and maintain network resilience as part of the Thread self-healing feature \cite{Thread_Group_Fundamentals}.
    \item \glspl{SED} are end devices that save energy by entering a low-power sleep mode when not actively communicating. \glspl{SED} can be present or absent in a network, but their inclusion helps optimize power consumption due to their energy-efficient sleep periods \cite{Thread_Group_Fundamentals}.
    \item In Thread networks, a leader router is always present, automatically elected through a decentralized process. This router manages network-wide configurations and operations, ensuring consistent performance and stability. Its constant presence supports the smooth functioning of the Thread network, adapting to changes in network topology or router failures \cite{Thread_Group_Fundamentals}.
    \item In a mesh network, at least two routers are required to establish connectivity. However, having three or more routers, including a leader, greatly improves the network's resilience, redundancy, and coverage. Therefore, to create a more robust mesh network with a leader, it is recommended to have a minimum of three routers, with one of them serving as the leader. This configuration ensures enhanced network performance and maintains seamless communication throughout the network \cite{girgis2014solving}.
    \item Thread networks can have multiple border routers, with at least two present to prevent a single point of failure. This redundancy ensures continuous connectivity and communication between the Thread network and other \gls{IP}-based networks, maintaining network stability and reliability even if one border router encounters a failure \cite{Thread_Group_Fundamentals}.
\end{enumerate}

The following mathematical model is designed for this purpose \cite{girgis2014solving}:

\begin{equation}\label{eq:minimize_power}
    \begin{aligned}
        Min\sum_{i=1}^{M}P_{tx}^i
    \end{aligned}
\end{equation}

\textbf{Subjects to:}
\begin{equation}\label{eq:mathematical_constraints}
    \begin{split}
        RSSI_{Device}^j>Sensitivity,j\in1,\cdots,N \\
        -20dBm{\le P}_t^j\le8dBm \\
        N_{REED}=N_{Router}+N_{Leader} \\
        SED\in0,1 \\
        N_{Leader}=1 \\
        N_{Router}+N_{Leader}\geq3 \\
        N_{BR}=2
    \end{split}
\end{equation}

Where $P_{tx}$ represents the transmit power $(dBm)$ of each one of the $M$ devices and $N$ is amount of devices.

\section{\texorpdfstring{\acrlong{MCM}}{MCM} Process}\label{sec:monte_carlo_method_process}

The \gls{MCM} involves four main steps. First, the process is initialized with predefined parameters and constraints. Second, random numbers are generated within the defined bounds to explore various network configurations. Third, the generated configurations are evaluated based on their performance and adherence to constraints. Finally, after a predetermined number of iterations or reaching an acceptable solution, the \gls{MCM} process comes to an end, providing an optimized network configuration. For a detailed explanation of each step, refer to the respective sections below.

\subsection{Initialize}

The \gls{MCM} is initiated to optimize the Thread network, considering the key parameters influencing the network's performance and energy efficiency. These parameters are outlined in the table below:

\begin{longtable}{>{\hspace{0pt}}m{0.098\linewidth}>{\hspace{0pt}}m{0.842\linewidth}}
    \label{tab:monte_carlo_parameters}\\
    \caption{Parameters influencing \acrlong{MCM}.}\\
    \hline\hline
    Param & Description
    \endfirsthead
    \hline
    $N_d$     & The total number of devices participating in the network, which is set to 8 for this research, representing a small-scale \gls{IoT} network.                                                                                                             \\
    \hline
    $P_{tx}$  & Determines the signal strength for each device, randomly generated in a range between -20 $dBm$ and 8 $dBm$ according to the mathematical constraints, affecting network connectivity and energy consumption \cite{semiconductor_nrf52840_2018_1}.       \\
    \hline
    $F_c$     & The carrier frequency used for calculating \gls{RSSI} using the general path loss model, set at 2.4 ${GH}_z$, based on Thread protocol specification \cite{Thread_Group_Fundamentals}.                                                                   \\
    \hline
    $D_0$     & A reference distance of 0.25 $m$, associated with the carrier frequency $F_c$, employed in the log-distance path loss model in equation \ref{eq:log_distance_path_loss} to calculate the signal attenuation \cite{cho2010mimo}.                          \\
    \hline
    $d$       & Represents the distance between two devices in the network, as illustrated in figures \ref{fig:distance_matrix_lab} and \ref{fig:distance_matrix_home}, influencing the strength of the signal received by devices.                                      \\
    \hline
    $n$       & The path loss exponent shown in equation \ref{eq:log_distance_path_loss}, set to 5.0, which represents the rate at which the signal power decays with distance in the path loss model \cite{cho2010mimo}.                                                \\
    \hline
    $\sigma$  & The variance of the shadowing component, set to 3.0 $dB$, accounts for signal fluctuations due to obstacles and multipath propagation in the environment as shown in equation \ref{eq:log_normal_shadowing_model} \cite{cho2010mimo}.                    \\
    \hline
    $G_t$     & The transmit antenna gain, set to 0.0 $dB$, which reflects the effectiveness of the transmitting antenna in directing the radio waves towards the receiving device \cite{semiconductor_nrf52840_2018_1}.                                                 \\
    \hline
    $G_r$     & The receive antenna gain, set to 0.0 $dB$, indicating the receiving antenna's ability to capture incoming radio waves \cite{semiconductor_nrf52840_2018_1}.                                                                                              \\
    \hline\hline
\end{longtable}

\subsection{Generate Random Numbers}

Based on the factors mentioned at the start, \gls{MCM} generates a vector $X$ of length equal to $2n$, where $n$ is the number of places where network elements can be allocated \cite{1576539}. The vector is represented as:

\begin{equation}\label{eq:vector_x}
    \begin{split}
        X=\left[x_1,x_2,x_3,\cdots x_n,p_1,p_2,p_3,\cdots,p_n\right] \\
        for{\ x}_n\in0,1,2,3,4,5 \\
        p_n\in-20:4:8\ dBm
    \end{split}
\end{equation}

    Where $0$ represents no element allocated, $1$ is allocate as a \gls{SED}, $2$ is allocate a \gls{REED}, $3$ is allocate a router, $4$ is allocate the leader, and $5$ is allocate a border router.
\vspace{2mm}

\subsection{Evaluate Results}

The objective function aims to build a Thread network using the optimal network configuration without violating the mathematical constraints. If a constraint is violated, a penalty is added to the objective function, which is weighted according to the importance of the constraint. The objective function with penalty values can be written as:

\begin{equation}\label{eq:objective_function}
    \begin{aligned}
        Min\sum_{i=1}^{M}P_{tx}^i+penal_1+penal_2+penal_3+penal_{nr}
    \end{aligned}
\end{equation}

    Where $penal_1$ represents penalty for violating the first restriction, $penal_2$ is penalty for violating the second restriction, and $penal_{nr}$ is the penalty for violating the last restriction.
\vspace{2mm}

\subsection{Termination}

The \gls{MCM} converges on an optimal solution that satisfies necessary constraints, providing outputs such as device types, transmission power, and position. It also offers information on constraint violations, including the penalty, power consumption, and \gls{RSSI} sensitivity violations—these outputs aid in understanding the optimization process and refining the network design. For a comprehensive understanding of the four steps of the \gls{MCM} process, refer to the following pseudocode, which provides an overview of the algorithm's structure and logic.

\begin{algorithm}[H]
    \caption{\gls{MCM} pseudocode for network optimization.}
    \label{alg:mcm}
    \begin{algorithmic}
    \STATE Initialize \gls{MCM} parameters: $N_d$, $d$, $F_c$, $D_0$, $n$, $\sigma$, $G_t$, $G_r$
    \WHILE{network}
        \STATE devices, txpower, position $\gets$ generate\_random\_numbers($N_d$)
        \STATE penalty, path\_loss, rssi $\gets$ mathematical\_constraints\_evaluation($N_d$, $d$, $F_c$, $D_0$, $n$, $\sigma$, $G_t$, $G_r$)
        \IF{penalty is False}
            \STATE network $\gets$ False
        \ENDIF
        \RETURN devices, txpower, penalty, path\_loss, rssi
    \ENDWHILE
    \end{algorithmic}
\end{algorithm}

It is a simplified version of the \gls{MCM} implementation and does not cover all the details of the original code. It is focused on the primary structure and steps of the method for network optimization and initial network build-up transmission power. To access the complete version of the algorithm code, including all implementation details, refer to the appendix \ref{appendix} section.

\section{\texorpdfstring{\acrlong{GA}}{GA} Process}\label{sec:genetic_algorithm}

The \gls{GA} process can be summarized into four main steps: initializing population, evaluating fitness, performing selections, and finding the best solution. These steps are designed to optimize transmission power in the network by evolving a population of candidate solutions through generations. In the following paragraphs, each step is discussed in detail.

\subsection{Initialize}

The initial steps of the \gls{GA} process start with creating a random population with the specified population size, representing different possible network configurations. The population is generated based on the parameters set, as shown below:

\begin{longtable}{>{\hspace{0pt}}m{0.077\linewidth}>{\hspace{0pt}}m{0.865\linewidth}}
    \label{tab:ga_parameters}\\
    \caption{Parameters influencing \gls{GA}.}\\
    \hline\hline
    Param            & Description \endfirsthead
    \hline
    Population size  & The number of individuals in the population representing different possible network configurations are set to 100 for this research.                                                                                                                                                                                                                             \\
    \hline
    Population       & An initial random population is created with the specified population size and \gls{MCM} output, which includes device types, transmission power, and device positions, representing different possible network configurations. For instance: [[device: 3, 5, 2, 5, 1, 5, 0, 0], [position: 1, 2, 3, 4, 5, 6, 7, 8], [txpower: -20, 0, 0, -8, 0, -12, 0, -20]].  \\
    \hline
    Max iteration    & The maximum number of iterations to be performed by the \gls{GA}, for instance, 100 in this research.                                                                                                                                                                                                                                                            \\
    \hline
    Mutation rate    & The probability of mutation is set at 0.1 for this research, determining the frequency of random changes introduced to the offspring's genetic information during the optimization process, which helps maintain genetic diversity within the population.                                                                                                        \\
    \hline
    Selection method & The method used for selecting individuals from the current population to create the next generation, such as roulette wheel selection, tournament, or sorted. In this research, the sorted selection method was utilized.                                                                                                                                        \\
    \hline
    Mutation method  & The method used for mutating individuals affect how genetic information is altered during the mutation process. In this research, the swap mutation method was utilized.                                                                                                                                                                                         \\
    \hline\hline
\end{longtable}

\subsection{Evaluate Fitness}

In the fitness evaluation stage of the \gls{GA}, each individual in the population is evaluated based on a fitness function. This fitness function is responsible for computing the fitness score of each individual, which, in this context, is represented as a chromosome \cite{lambora2019genetic}. Each chromosome in the population is composed of a list representing different device types, their respective positions, transmission powers, and an initially assigned penalty score of zero. For instance, here are some examples of chromosomes in the population:

\begin{enumerate}
    \item \textbf{Chromosome 1}: [[device: 2, 5, 3, 3, 5, 2, 2, 4], [position: 1, 2, 3, 4, 5, 6, 7, 8], [txpower: -8, 8, 0, -16, 0, -8, -20, 8], [penalty: 0]]
    \item \textbf{Chromosome 2}: [[device: 2, 5, 3, 3, 5, 2, 2, 4], [position: 1, 2, 3, 4, 5, 6, 7, 8], [txpower: -12, 4, -8, -8, -20, 0, -8, -20], [penalty: 0]]
    \item \textbf{Chromosome 3}: [[device: 2, 5, 3, 3, 5, 2, 2, 4], [position: 1, 2, 3, 4, 5, 6, 7, 8], [txpower: 4, -12, -20, -4, -8, -12, -20, -8], [penalty: 0]]
\end{enumerate}

For each chromosome, the fitness function calculates the path loss and \gls{RSSI} sensitivity for each device configuration. The fitness function ensures that each chromosome adheres to the established mathematical constraints. When a chromosome violates a constraint, a penalty is added to the penalty score of that chromosome. The calculation and addition of penalties occur after the evaluation of each constraint, adjusting the fitness value accordingly. The fitness value, consequently, provides a measure of the quality of each solution, with lower penalties indicating more desirable solutions.

\subsection{Selection}

In the selection process, the sorted method is employed to identify the fittest individuals in the current population \cite{lambora2019genetic}. The entire population is sorted based on their penalties for constraint violations. The most fit individuals are then selected and stored in a separate list. Let's consider the following three chromosomes excluding device types and positions:

\begin{enumerate}
    \item \textbf{Chromosome 1}: [[txpower: -8, 8, 0, -16, 0, -8, -20, 8], [penalty: 0]]
    \item \textbf{Chromosome 2}: [[txpower: -12, 4, -8, -8, -20, 0, -8, -20], [penalty: 4000]]
    \item \textbf{Chromosome 3}: [[txpower: 4, -12, -20, -4, -8, -12, -20, -8], [penalty: 3000]]
\end{enumerate}

The sorted list of chromosomes will look like this:

\begin{enumerate}
    \item \textbf{Chromosome 1}: [[txpower: -8, 8, 0, -16, 0, -8, -20, 8], [penalty: 0]]
    \item \textbf{Chromosome 3}: [[txpower: 4, -12, -20, -4, -8, -12, -20, -8], [penalty: 3000]]
    \item \textbf{Chromosome 2}: [[txpower: -12, 4, -8, -8, -20, 0, -8, -20], [penalty: 4000]]
\end{enumerate}

Chromosome 1, having the lowest penalty, is the fittest chromosome. This approach ensures that the algorithm focuses on the most promising solutions as it progresses through the crossover and mutation stages.

\subsection{Crossover}

The crossover operation utilizes genetic material from the output of the selection method to form two parent solutions, subsequently creating offspring that inherit properties from both parents. This process aims to explore new potential solutions in the search space by allowing offspring to possess a mix of characteristics from their parents \cite{lambora2019genetic}. For instance, consider the following parent chromosomes selected from the previous selection process:

\begin{enumerate}
    \item \textbf{Chromosome 1}: [[txpower: \textcolor{red}{-8, 8, 0, -16, 0, -8, -20, 8}], [penalty: 0]]
    \item \textbf{Chromosome 2}: [[txpower: \textcolor{blue}{ 4, -12, -20, -4, -8, -12, -20, -8}], [penalty: 3000]]
\end{enumerate}

A random crossover point is determined within the length of the parent solutions, say at the fourth index for this instance. Consequently, offspring are generated by merging the first part of Parent Chromosome 1 up to the crossover point with the second part of Parent Chromosome 2 from the crossover point onwards, and vice versa. This results in the following offspring:

\begin{enumerate}
    \item \textbf{Chromosome 1}: [[txpower: \textcolor{red}{-8, 8, 0, -16,} \textcolor{blue}{-8, -12, -20, -8}], [penalty: TBD]]
    \item \textbf{Chromosome 2}: [[txpower: \textcolor{blue}{4, -12, -20, -4,} \textcolor{red}{0, -8, -20, 8}], [penalty: TBD]]
\end{enumerate}

Here, ``TBD'' indicates that the penalty for each offspring chromosome will be determined in subsequent processes based on the updated txpower values.

Through this method, two new offspring are produced, each inheriting distinct portions of the parent solutions. This mechanism potentially leads to enhanced solutions in subsequent generations, thereby ensuring the evolution of the population towards optimal solutions.

\subsection{Mutation}

The mutation stage further enhances the diversity within the population, ensuring a thorough exploration of the search space. Offspring from the crossover stage serve as the input to the mutation operation \cite{lambora2019genetic}. For instance, consider the following offspring solutions:

\begin{enumerate}
    \item \textbf{Chromosome 1}: [[txpower: -8, 8, 0, -16, -8, -12, -20, -8], [penalty: TBD]]
    \item \textbf{Chromosome 2}: [[txpower: 4, -12, -20, -4, 0, -8, -20, 8], [penalty: TBD]]
\end{enumerate}

A mutation rate controls the probability of mutation for each element within the offspring solution. If a random value, obtained through a uniform distribution, is less than the mutation rate, the respective element undergoes mutation \cite{lambora2019genetic}. In this instance, a mutation could be a change in the txpower value. Let's say the fourth element of Offspring Chromosome 1 and the second element of Offspring Chromosome 2 are selected for mutation. The txpower values may then be adjusted, resulting in the following mutated offspring:

\begin{enumerate}
    \item \textbf{Chromosome 1}: [[txpower: -8, 8, 0, \textcolor{red}{-10}, -8, -12, -20, -8], [penalty: TBD]]
    \item \textbf{Chromosome 2}: [[txpower: 4, \textcolor{blue}{-10}, -20, -4, 0, -8, -20, 8], [penalty: TBD]]
\end{enumerate}

Here, ``TBD'' also implies that the penalty for each mutated offspring chromosome will be recalculated in subsequent processes based on the new txpower values.

This mutation process ensures that the offspring solutions can investigate different combinations of transmission powers, which could potentially yield superior overall solutions in future generations.

\subsection{Population Update}

The algorithm iteratively refines the population by applying the selection, crossover, and mutation operations in each generation. After the offspring are created through crossover and mutation, their fitness values are calculated again. The population is then updated by replacing the current individuals with the new offspring, sorted based on their fitness values. This process of updating the population ensures that the best solutions are carried forward to the next generation \cite{lambora2019genetic}.


\subsection{Termination}

The algorithm continues this process of generating new offspring and updating the population for a specified number of iterations. Once the termination criterion is met, the algorithm concludes, and the final population along with their fitness values are returned as output \cite{lambora2019genetic}. The best solution can be extracted from this final population, representing the optimal device types, optimized transmission power, and positions for each device in the given problem scenario. The following pseudocode gives an overview of the \acrlong{GA}process:

\begin{algorithm}[H]
    \caption{\gls{GA} pseudocode for transmission power optimization.}
    \label{alg:genetic_algorithm}
    \begin{algorithmic}[1]
    \STATE Initialize \gls{GA} parameters: population\_size, population, max\_iterations, mutation\_rate, selection\_method
    \STATE Initialize population: create\_random\_population(population\_size)
    \FOR{each candidate in population}
        \STATE fitness = evaluate\_fitness(candidate)
    \ENDFOR
    \FOR{generation in range(max\_iterations)}
        \STATE parents = select\_parents(population, selection\_method)
        \STATE offspring = crossover(parents, crossover\_prob)
        \STATE offspring = mutate(offspring, mutation\_prob)
        \FOR{each candidate in offspring}
            \STATE fitness = evaluate\_fitness(candidate)
        \ENDFOR
        \STATE population = replace\_population(population, offspring)
    \ENDFOR
    \STATE best\_solution = find\_best\_solution(population)
    \end{algorithmic}
\end{algorithm}

The output is a list of optimized transmission power values for each device, along with optimal device types, and positions. For a comprehensive understanding of the \acrlong{GA}'s implementation, refer to the appendix \ref{appendix} for the complete code.

\section{Experimental Setup}

The prototype was built to validate the output from \gls{MCM} and \gls{GA}, using the optimal network configuration determined by \gls{MCM}, which consisted of a total of 8 devices. The setup included 2 border routers, 3 routers (with one of them automatically elected as a leader), and 3 \glspl{REED}. The prototype was designed to closely resemble the conceptual model presented earlier in the figure, with the only slight difference being the use of \glspl{REED} instead of sensors as the end devices. An image was provided below to illustrate the Thread network topology that had been constructed.

\begin{figure}[H]
    \centering
    \includegraphics[width=1\textwidth]{images/research_design/prototype.png}
    \caption{Thread network topology of the prototype.}
    \label{fig:prototype}
\end{figure}

The construction process of the prototype involved several crucial steps, aimed at validating the output from \gls{MCM} and \gls{GA} and ensuring optimal network configuration:

\begin{enumerate}
    \item Customized \gls{nRF} Thread Client and Server \gls{SDK} to fit the needs for the research, selecting roles for each device and setting the transmission power output from both \gls{MCM} and \gls{GA} for optimal network configuration.
    \item Flashed each router with the Thread Server and each \gls{REED} with the Thread Client customized \gls{SDK}. In this configuration, routers acted as servers, while \glspl{REED} acted as clients. Communication between devices was bidirectional, with the clients having \gls{BLE} enabled for multiprotocol support.
    \item Flashed the border router nodes with the Coprocessor setup provided by \gls{nRF}. To enable the Raspberry Pi to act as an edge device, the OpenThread \gls{RCP} architecture was implemented.
    \item Turned on the devices one by one, noting that the first device activated in the network was most likely to become the leader, although leadership could change during the network's lifetime.
    \item Validated all the nodes by running multicast messages using Thread \gls{ICMP} service. The \gls{ICMP} service allowed sending echo requests (ping) to devices, activating their Thread antennas. This enabled testing the Thread connection, and devices could also reply \cite{Thread_Group_Fundamentals}.
    \item Validated the multiprotocol support connection by running a data flow from the ESP32 \gls{UWB} and mobile devices to the \glspl{REED} through \gls{BLE}, which then forwarded the data to the routers. This step ensured seamless communication between non-Thread devices and the Thread network.
    \item Monitored the network for stability and performance, adjusting settings to maintain optimal operation.
\end{enumerate}

Following these steps, the prototype was successfully constructed to apply the optimized settings obtained from the \gls{MCM} and \gls{GA}. The subsequent figure presented a real-world Thread network prototype setup from the lab setup. The image provided a clear view of the \gls{nRF}52840-based Thread nodes, Raspberry Pi as the edge device, and the border router setup with the dongle. It also showcased the development kits used for routers and \glspl{REED}.

\begin{figure}[H]
    \centering
    \includegraphics[width=0.8\textwidth]{images/research_design/prototype_setup.jpg}
    \caption{Thread network prototype setup in the lab.}
    \label{fig:prototype_setup}
\end{figure}

\section{Data Collection}

The data collection process aimed to validate and compare the solutions from \gls{MCM} and \gls{GA} against the maximum mode by measuring power consumption in each device of the built prototype. Utilizing the \gls{nRF} \gls{PPK} II, which offers Current $\left(ma\right)$ measurement at 100,000 samples per second, allowed for accurate power consumption measurements across various scenarios, locations, network activities, optimization modes, and durations. This approach provided insights into the effectiveness of the optimization techniques in both controlled and real-world settings while avoiding excessive data that would have complicated the analysis process.

\begin{enumerate}
    \item \textbf{Method}: Power consumption was measured in two primary scenarios - Maximum and Optimized. The maximum scenario represented the baseline power consumption, where no optimization techniques were applied. The optimized scenario measured power consumption after implementing the \gls{MCM} and \gls{GA} optimization techniques.
    \item \textbf{Location}: Measurements were conducted in two different locations - Lab and Home. The lab setting, smaller in size compared to the home location, allowed for controlled environments and reproducible results. The home setting provided a real-world context with a larger area, helping to understand the performance of the Thread network in everyday \gls{IoT} applications. The following images show the Euclidean distance matrix from two different locations to share a clear view of the distance between each device in the two locations.
    \begin{figure}[H]
        \centering
        \begin{minipage}{0.5\textwidth}
            \centering
            \includegraphics[width=1\textwidth]{images/research_design/distance_matrix_lab.png}
            \captionof{figure}{Distance matrix for lab.}
            \label{fig:distance_matrix_lab}
        \end{minipage}%
        \begin{minipage}{0.5\textwidth}
            \centering
            \includegraphics[width=1\textwidth]{images/research_design/distance_matrix_home.png}
            \captionof{figure}{Distance matrix for home.}
            \label{fig:distance_matrix_home}
        \end{minipage}
    \end{figure}
    \item \textbf{Type}: Power consumption measurements were also conducted based on the type of network activity. The no sensor scenario represented a Thread network with no active sensors, while the ping scenario simulated data exchange between nodes, resembling real \gls{IoT} network behavior \cite{Thread_Group_Fundamentals}.
    \item \textbf{Mode}: The project compared the effectiveness of \gls{MCM} and \gls{GA} optimization techniques. The \gls{MCM} mode measured power consumption based on network configurations optimized using the \acrlong{MCM}. The \gls{GA} mode measured power consumption with network configurations optimized using the \acrlong{GA}.
    \item \textbf{Duration}: Power consumption measurements were conducted for different durations - 60 seconds in the lab location and 300 seconds in the home location. This variation in duration helped in understanding the impact of time on power consumption in different environments.
    \item \textbf{Ping}: In the lab location, 50 pings were sent within the 60-second duration, whereas in the home location, 290 pings were sent during the 300-second duration. This distinction helped analyze the impact of network activity on power consumption in both controlled and real-world settings.
\end{enumerate}

Following the data collection steps, two images are provided to illustrate the process of collecting power consumption data from the \gls{nRF}52840 \gls{DK} using the \gls{PPK} II. These images offer a visual representation of the setup and the data collection process, giving a clearer understanding of the experimental context and the methods used for obtaining the power consumption measurements.

\begin{figure}[H]
    \centering
    \begin{minipage}[t]{0.45\textwidth}
        \centering
        \includegraphics[width=1\linewidth]{images/research_design/PPK2_Router.jpg}
        \captionof{figure}{\gls{PPK} II connected to a router.}
        \label{fig:router_source_meter}
    \end{minipage}\hfill
    \begin{minipage}[t]{0.45\textwidth}
        \centering
        \includegraphics[width=1\linewidth]{images/research_design/PPK2_SDK.jpg}
        \captionof{figure}{\gls{PPK} II software in source meter mode.}
        \label{fig:ppk2_source_meter}
    \end{minipage}
\end{figure}
