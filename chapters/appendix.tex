\chapter*{Appendix}\label{appendix}
% Following is required as above uses \chapter*{} (note the star). The start makes the chapter unnumbered, but also removes it from table of content. Former is desired, the latter is not:
\addcontentsline{toc}{chapter}{Appendix}


% Temporarily remove the chapter number from the section counter
\counterwithout{section}{chapter}


\appendix
This Appendix provides various resources and links related to the research project. These resources include the full dataset analysis, algorithms, custom implementations, and output datasets. Due to these materials' large size and complexity, it is not feasible to include them directly in the research paper. Instead, the links in the following sections grant access to the complete datasets, algorithms, and implementations, allowing interested readers to explore the project in greater detail and better understand the methodology, optimization techniques, and findings. The sections below outline the resources available in the Appendix.

\section{Dataset Analysis}
This section provides the link to the Dataset Analysis Repository on GitLab. This repository contains a comprehensive set of analyses performed for the project. Due to the extensive nature of the analyses, including them all in this paper is not feasible. By sharing the repository, readers can access detailed studies and better understand the project's intricacies. The repository can be accessed using the following link: \url{https://gitlab.com/mmikhan/threadpowerprofiler/}

\section{Dataset}
The complete dataset, too large to include within the research paper, is available on GitLab. This dataset contains detailed information on the performance of the Thread network under various conditions and configurations. The original dataset is in binary format but has been converted to CSV for convenience and easier access. Access the dataset here: \url{https://gitlab.com/mmikhan/threadpowerprofiler/}

\section{Algorithm}
This section links the complete algorithm consisting of the Monte Carlo Method and Genetic Algorithm implementations on GitHub. This repository houses the code and documentation required to understand and replicate the optimization techniques used in this research project. Access the algorithm here: \url{https://github.com/mmikhan/ThreadNetPowerOptGA}

\section{nRF Thread Client and Server Custom Implementation}
This part presents the custom implementation of the nRF Thread Client and Server used in the physical prototype. This implementation was essential to successfully deploying and testing the optimized Thread network. Access the nRF Thread Client and Server custom implementation here: \url{https://github.com/mmikhan/Connecta}

\section{Optimization Results}
Finally, this section provides access to the large output dataset from Monte Carlo Method and Genetic Algorithm simulations. This dataset is crucial for understanding the outcomes of the optimization techniques and their impact on the energy efficiency and performance of the Thread network. Access the output dataset here: \url{https://gitlab.com/mmikhan/threadpowerprofiler/}

\vspace{2mm}
The resources presented in the Appendix thoroughly examine the research project, its methodology, and the optimization techniques utilized. Through carefully studying these materials, a comprehensive understanding of the project's development, implementation, and outcomes can be obtained, thereby enriching the overall context of the research.


% Restore the chapter number to the section counter
\counterwithin{section}{chapter}
