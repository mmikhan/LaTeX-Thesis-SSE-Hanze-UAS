\chapter{Situational \& Theoretical Analysis}\label{chap:situational_theoretical_analysis}

\section{The Thread Protocol}\label{sec:the_thread_protocol}
Thread is a low-power, wireless \gls{IoT} protocol designed to provide secure, reliable, and scalable networking for connected devices. Developed by the Thread Group, which includes notable members such as Nest Labs (a subsidiary of Google), \gls{ARM}, and Silicon Labs, Thread was introduced in 2014 to address the growing need for a standardized and efficient \gls{IoT} networking solution. Built on open standards, Thread is an \gls{IPv6} based protocol that utilizes the \gls{IEEE} 802.15.4 radio standard for communication, making it compatible with a wide range of existing devices and technologies \cite{Thread_Group_Fundamentals}.


\subsection{Architecture and Components}
Thread's architecture is based on a mesh topology, allowing devices to communicate directly with each other, bypassing the need for a central hub or router. This mesh design enhances network resilience, as devices can automatically re-route communication through alternative paths if a connection is lost \cite{Thread_Group_Fundamentals}. The key components of the Thread protocol include:

\begin{enumerate}
    \item \textbf{Border Routers}: These devices serve as gateways between the Thread network and external \gls{IP} networks, such as Wi-Fi or Ethernet networks. They manage network access, security, and routing of data between the Thread network and other networks \cite{Thread_Group_Fundamentals}.
    \item \textbf{Leader Routers}: They play a vital role in managing the network by assigning addresses to devices, coordinating routing updates, and maintaining overall network stability \cite{Thread_Group_Fundamentals}.
    \item \textbf{Routers}: These devices are responsible for routing data within the Thread network. They can also act as parent devices to other devices within the network, providing connectivity to devices with limited routing capabilities \cite{Thread_Group_Fundamentals}.
    \item \textbf{End Devices}: These devices communicate directly with their parent routers and are typically low-power devices, such as sensors or actuators. End devices do not participate in routing or network management \cite{Thread_Group_Fundamentals}.
    \item \textbf{Links}: Thread networks use links to establish connections between devices, allowing them to communicate and exchange data. Links are essential for maintaining the mesh topology of Thread networks \cite{Thread_Group_Fundamentals}.
\end{enumerate}

Figure \ref{fig:thread_topology} shows a visual representation of the basic Thread network topology, which includes all the listed components \cite{Thread_Group_Fundamentals}. By combining these components, the Thread network architecture provides a robust, scalable, and energy-efficient solution for \gls{IoT} applications, including the MOOD-Sense project.

\begin{figure}[H]
    \centering
    \includegraphics[width=0.8\textwidth]{images/situational_theoretical_analysis/thread_topology.png}
    \caption{Basic Thread network topology \cite{Thread_Group_Fundamentals}.}
    \label{fig:thread_topology}
\end{figure}


\subsection{Key Features and Advantages}

Considering the specific requirements of various \gls{IoT} applications, including the MOOD-Sense project, the following features and advantages of Thread make it a suitable choice:

\begin{enumerate}
    \item \textbf{Low Power Consumption}: Thread's energy-efficient design aligns with the need for long battery life in devices that continuously monitor, collect, and transmit data in various \gls{IoT} scenarios \cite{Thread_Group_Fundamentals}.
    \item \textbf{Scalability}: The mesh topology of Thread networks allows for the seamless addition of new devices, enabling \gls{IoT} projects to adapt and expand as needed \cite{Thread_Group_Fundamentals}.
    \item \textbf{Security}: Thread's end-to-end encryption and secure commissioning processes ensure that communication between devices is protected, maintaining data privacy and security across diverse applications \cite{Thread_Group_Fundamentals}.
    \item \textbf{Robustness and Reliability}: Thread's self-healing mesh network design ensures reliable and resilient communication, which is crucial for continuous monitoring and data collection in \gls{IoT} applications \cite{Thread_Group_Fundamentals}.
    \item \textbf{Interoperability}: Thread's open standards ensure compatibility with a wide range of devices and technologies, allowing \gls{IoT} projects to integrate various sensors, devices, and communication technologies within a single, unified network \cite{Thread_Group_Fundamentals}.
\end{enumerate}

Overall, the Thread protocol's features address the key question of its suitability for \gls{IoT} applications in various contexts, including the MOOD-Sense project. Offering a secure, reliable, and energy-efficient networking solution, Thread meets the requirements for continuous monitoring, improved data collection, and seamless integration across industries.


\section{Power Optimization}\label{sec:power_optimization}

Optimizing power consumption in wireless \gls{IoT} networks is a critical challenge, particularly for applications like the MOOD-Sense project, where devices are expected to operate for extended periods without frequent battery replacements or recharging. One effective approach to reduce power consumption is by minimizing transmission power while still maintaining reliable communication among devices, taking into account factors such as path loss and signal strength \cite{sheth2002implementation}. This section describes the implementation of transmission power control for Thread wireless networks, aiming to optimize the overall power consumption and ensure efficient and reliable communication among devices within the MOOD-Sense project context.

\subsection{Factors Influencing Transmission Power}

Parameters influencing transmission power in a Thread network are crucial for optimizing energy efficiency and network performance. By examining these factors, network designers can make informed decisions to achieve optimal performance under various conditions. A thorough understanding of these parameters is essential for implementing effective power management strategies and maintaining reliable communication within the network. Some of the factors include:

\subsubsection{Distance}

The distance between devices directly influences the transmission power, as a longer distance between devices typically results in higher path loss \cite{wu2014study}. Therefore, devices that are farther apart may require higher transmission power levels to maintain a stable connection. The Euclidean distance matrix calculates the distance between pairs of devices \cite{dokmanic2015euclidean}. The distance between devices $i$ and $j$ with coordinates $\left(x1,y1\right)$ and $\left(x2,y2\right)$ is calculated as follows:

\begin{equation}\label{eq:euclidean_distance}
    distance\left(i,j\right)=\sqrt{\left(x^2-x^1\right)^2+\left(y^2-y^1\right)^2}
\end{equation}

This calculation helps account for spatial constraints and device placements, ensuring \gls{MCM} random input generation considers these factors for a more efficient and optimized Thread network \cite{dokmanic2015euclidean}.

\subsubsection{\acrlong{RSSI}}

\acrfull{RSSI} is a measurement of the power level of a received radio signal. It helps to determine the link quality between devices in a wireless network. A higher \gls{RSSI} value indicates a stronger received signal, which may require lower transmission power to maintain reliable communication \cite{benkic2008rssi}. The \gls{RSSI} calculation, including transmit and receive antenna gains, is:

\begin{equation}\label{eq:rssi}
    RSSI=P_t+G_t+G_r-L_p
\end{equation}

Where $P_t$ is the transmission power $\left(dBm\right)$, $G_t$ is the transmit antenna gain $\left(dBi\right)$, $G_r$ is receive antenna gain $\left(dBi\right)$, and $L_p$ is path loss $\left(dB\right)$ \cite{doi:10.1155/2014/371350}.

Thread devices typically have an \gls{RSSI} sensitivity of -100 $dBm$. This formula applies to uplink and downlink connections, offering a more accurate signal strength representation and aiding in network performance optimization and energy consumption \cite{semiconductor_nrf52840_2018_1}.

\subsubsection{Antenna Gain}

The gain of the antennas used in the network can also affect the transmission power. A higher gain antenna can focus the radio signal more effectively, requiring less transmission power to achieve the same signal strength at the receiver \cite{wu2014study}.

\subsubsection{Path Loss}

Path loss refers to the attenuation of the radio signal as it propagates through the environment, depending on factors such as distance, frequency, and environmental conditions \cite{cho2010mimo}. It significantly impacts the transmission power needed for reliable communication. In this research, the log-normal shadowing model is used for path loss calculations. To understand this model, the three key path loss models are discussed, providing insight into the principles and mathematical formulations involved in path loss calculations for power optimization.

\begin{enumerate}
    \item \textbf{Free-Space Propagation Model}: The free-space propagation model is used for predicting the received signal strength in \gls{LOS} environments, where there are no obstacles between the transmitter and receiver. It is often adopted for satellite communication systems. The Friis equation \ref{eq:friis_equation} describes the received power at distance $d$, considering non-isotropic antennas with transmit gain $G_t$ and receive gain $G_r$ \cite{cho2010mimo}:

    \begin{equation}\label{eq:friis_equation}
        P_r\left(d\right)=\frac{P_tG_tG_r\lambda^2}{\left(4\pi\right)^2d^2L}
    \end{equation}

    Where $P_t$ represents the transmit power $\left(w\right)$, $d$ is the distance between transmitter and receiver $\left(m\right)$, $\lambda$ is the wavelength of radiation $\left(m\right)$, $G_t$ is transmit gain $(dB)$, $G_r$ receive gain $(dB)$, and $L$ is the system loss factor independent of the propagation environment. The free-space path loss ${PL}_F\left(d\right)$ can be directly derived without any system loss from equation \ref{eq:friis_equation}:

    \begin{equation}\label{eq:free_space_path_loss}
        PL_F\left(d\right)\left[dB\right]=10log\left(\frac{P_t}{P_r}\right)=-10log\left(\frac{G_tG_r\lambda^2}{\left(4\pi\right)^2d^2}\right)
    \end{equation}

    Without antenna gains (i.e., $G_t=G_r=1$), equation \ref{eq:free_space_path_loss} is reduced to:

    \begin{equation}\label{eq:free_space_path_loss_no_antenna_gain}
        PL_F\left(d\right)\left[dB\right]=10log\left(\frac{P_t}{P_r}\right)=20log\left(\frac{4\pi d}{\lambda}\right)
    \end{equation}

    \item \textbf{Log-Distance Path Loss Model}: The log-distance path loss model is a more generalized approach, accounting for the varying path loss exponent $n$ depending on the environment. The path loss at distance $d$ is given by equation \ref{eq:log_distance_path_loss}, where $d_0$ is the reference distance at which the path loss inherits the characteristics of free-space loss \cite{cho2010mimo}:

    \begin{equation}\label{eq:log_distance_path_loss}
        PL_{LD}\left(d\right)\left[dB\right]=PL_F\left(d_0\right)+10nlog\left(\frac{d}{d_0}\right)
    \end{equation}

    Where $d_0$ is a reference distance and $n$ corresponds to free space which tends to change as shown in the following table.

    \begin{longtable}{>{\hspace{0pt}}m{0.462\linewidth}>{\hspace{0pt}}m{0.467\linewidth}}
        \label{tab:path_loss_exponent}\\
        \caption{Path loss exponent for different environments.}\\
        \hline\hline
        Environment                   & Path Loss Exponent $(n)$  \endfirsthead
        \hline
        Free space                    & 2                         \\
        Urban area cellular radio     & 2.7 - 3.5                 \\
        Shadowed urban cellular radio & 3 - 5                     \\
        In building line-of-sight     & 1.6 - 1.8                 \\
        Obstructed in building        & 4 - 6                     \\
        Obstructed in factories       & 2 - 3                     \\
        \hline\hline
    \end{longtable}

    The path loss exponent $\left(n\right)$ varies based on the environment, as shown in table \ref{tab:path_loss_exponent}, and helps to adjust the log-distance path loss model for more accurate predictions. Lower values represent environments with fewer obstructions, such as free space, while higher values indicate more complex environments with buildings or other obstacles \cite{cho2010mimo}.

    \item \textbf{Log-Normal Shadowing Model}: The log-normal shadowing model considers the random nature of shadowing effects, making it more suitable for realistic situations. The model is given by equation \ref{eq:log_normal_shadowing_model}, where $X_\sigma$ is a Gaussian random variable with a zero mean and a standard deviation of $\sigma$ \cite{cho2010mimo}:

    \begin{equation}\label{eq:log_normal_shadowing_model}
        PL\left(d\right)\left[dB\right]=\overline{PL}\left(d\right)+X_\sigma=PL_F\left(d_0\right)+10nlog\left(\frac{d}{d_0}\right)+X_\sigma
    \end{equation}

    In other words, this particular model allows the receiver at the same distance $d$ to have a different path loss, which varies with the random shadowing effect $X_\sigma$ \cite{cho2010mimo}.
\end{enumerate}


\subsubsection{Device Types}

In a Thread network, devices can have different types and roles, such as routers, end devices, or border routers. These roles can impact the transmission power requirements, as routers may need to communicate with multiple neighboring devices, whereas end devices only need to communicate with their parent router \cite{Thread_Group_Fundamentals}.

\subsection{Algorithmic Approaches}

To optimize transmission power in a Thread network, algorithmic approaches can be employed. In this research, two algorithms are used to address transmission power optimization: the \gls{MCM} and the \gls{GA}. The \gls{MCM} focuses on selecting the right device types and configuring an optimal network based on the constraints and requirements of the MOOD-Sense project. The \gls{GA}, on the other hand, takes the output from the \gls{MCM} and optimizes the transmission power settings for each device, ensuring minimal energy consumption while maintaining network performance.

\subsubsection{\acrlong{MCM}}

The \acrfull{MCM} is a popular computational technique for simulating complex systems and estimating numerical results using random sampling. It is particularly useful for solving problems with a large number of variables and an extensive search space, where traditional analytical methods may be inefficient or infeasible \cite{kroese2014monte}. In the context of this research, \gls{MCM} is employed to generate and evaluate potential network configurations for optimizing power consumption. The key steps involved in the \gls{MCM} are as follows:

\begin{enumerate}
    \item \textbf{Problem Formulation}: Define the problem and its parameters, including the objective function, constraints, and variables \cite{kroese2014monte}. For this research, the goal is to establish a reliable Thread network configuration with optimal device types and roles while minimizing the initial power consumption of the entire network, ensuring network performance and reliability are maintained.
    \item \textbf{Sampling}: Generate a random sample of potential solutions to the problem within the search space. This involves randomly selecting values for the variables within their specified ranges \cite{kroese2014monte}. In the case of this research, the sample consists of different network configurations with various device types and roles, along with varying transmission power settings for each device in the network.
    \item \textbf{Evaluation}: Assess the quality of each sampled solution using the objective function and constraints defined in the problem formulation \cite{kroese2014monte}. In this research, the objective function is to achieve the optimal network configuration that ensures reliability and minimizes the initial power consumption of the entire network.
    \item \textbf{Termination}: Repeat the sampling and evaluation steps for a predefined number of iterations or until a convergence criterion is met, such as reaching a desired level of accuracy or observing no significant improvement over a number of iterations \cite{kroese2014monte}.
\end{enumerate}


\subsubsection{\acrlong{GA}}

The \acrfull{GA} is an optimization technique inspired by the process of natural selection. It uses a population of possible solutions and evolves them over time using genetic operators, such as mutation and crossover. The goal is to find an optimal solution for the given problem, such as transmission power in this research \cite{lambora2019genetic}. In a typical \gls{GA} process, the following steps are executed:

\begin{enumerate}
    \item \textbf{Initialization}: A population of candidate solutions is generated, either randomly or using a heuristic method. The population size is determined by a predefined parameter. Each individual in the population represents a potential solution to the problem \cite{norouzi2014genetic}. In this research, the output from \gls{MCM} is used as the initial population.
    \item \textbf{Fitness Evaluation}: The quality of each candidate solution in the population is assessed using a fitness function, which quantifies how well the individual solves the problem. The fitness value provides an objective measure to compare and rank individuals within the population \cite{norouzi2014genetic}. Each solution in the population is validated against the predefined constraints set forth in this research.
    \item \textbf{Selection}: Based on their fitness values, individuals are selected for reproduction. Selection methods, such as tournament selection or roulette wheel selection, are employed to choose the fittest individuals, favoring those with higher fitness values to participate in the creation of offspring for the next generation \cite{lambora2019genetic}. For instance, consider the following example where two binary format chromosomes are selected:
    \begin{enumerate}
        \item \textbf{Chromosome 1}: 1101100100110110
        \item \textbf{Chromosome 2}: 1101111000011110
    \end{enumerate}
    \item \textbf{Crossover}: This genetic operator combines the genetic information of two parent individuals to create offspring that inherit characteristics from both parents. Crossover promotes exploration of the search space and the exchange of beneficial traits between individuals \cite{lambora2019genetic}. Consider the following example:
    \begin{enumerate}
        \item \textbf{Parent Chromosome 1}: \textcolor{red}{11011} | \textcolor{red}{00100110110}
        \item \textbf{Parent Chromosome 2}: \textcolor{blue}{11011} | \textcolor{blue}{11000011110}
    \end{enumerate}
    After performing crossover operation in random manner, the offspring are:
    \begin{enumerate}
        \item \textbf{Offspring 1}: \textcolor{red}{11011} | \textcolor{blue}{11000011110}
        \item \textbf{Offspring 2}: \textcolor{blue}{11011} | \textcolor{red}{00100110110}
    \end{enumerate}
    In this case, the genetic information from the two parent chromosomes has been recombined to form the offspring chromosomes, which carry traits from both parents.
    \item \textbf{Mutation}: Mutation is another genetic operator that introduces small random changes in the genetic information of individuals. This process helps maintain diversity within the population and prevents the algorithm from getting stuck in local optima \cite{lambora2019genetic}. For instance, the following binary chromosomes can undergo mutation, where selected bits are randomly switched from 0 to 1, or vice versa:
    \begin{enumerate}
        \item \textbf{Original Parent Chromosome 1}: 1101100100110110
        \item \textbf{Original Parent Chromosome 2}: 1101111000011110
    \end{enumerate}
    After performing mutation operation in random manner, the offspring become:
    \begin{enumerate}
        \item \textbf{Mutated Offspring 1}: 110\textcolor{red}{0}111000011110
        \item \textbf{Mutated Offspring 2}: 11011\textcolor{red}{0}1000011\textcolor{red}{0}10
    \end{enumerate}
    In this example, the original parent chromosomes have been mutated to create the new offspring chromosomes. The mutation operation has introduced new characteristics into the offspring, helping to maintain genetic diversity in the population.
    \item \textbf{Replacement}: The least fit individuals in the population are replaced with the newly created offspring, ensuring the population size remains constant across generations \cite{lambora2019genetic}.
    \item \textbf{Termination}: The algorithm repeats the process of fitness evaluation, selection, crossover, mutation, and replacement for a predefined number of generations or until a convergence criterion is met, such as reaching a desired level of fitness or observing no significant improvement over a number of generations \cite{lambora2019genetic}.
\end{enumerate}

By combining \gls{MCM} and \gls{GA}, the proposed algorithmic approach efficiently explores the parameter space and identifies the optimal configuration that minimizes power consumption while maintaining network performance and reliability in the Thread network. Addressing the sub-research question 3, this research demonstrates the effectiveness of the combined approach in achieving the desired outcomes.


\subsection{Step-by-Step Guide for Power Optimization}

\begin{enumerate}
    \item \textbf{Identify the Influencing Factors and Parameters}: Determine the parameters that impact transmission power, such as distance between devices, path loss, signal strength, interference, and device types.
    \item \textbf{Develop the Algorithms}: Implement the \gls{MCM} and \gls{GA} algorithms to address power optimization in the Thread network, considering the constraints and requirements of the MOOD-Sense project.
    \item \textbf{Determine Optimal Device Types and Network Configuration}: Use the \gls{MCM} to identify the appropriate device types and optimal network configuration for the Thread network.
    \item \textbf{Optimize Transmission Power Settings}: Apply the \gls{GA} to optimize the transmission power settings for each device in the network, ensuring minimal energy consumption while maintaining reliable communication.
    \item \textbf{Evaluate Network Performance}: Assess the overall performance of the Thread network, ensuring that the optimized power settings do not compromise the network's reliability or efficiency.
    \item \textbf{Iterate and Refine}: Continuously refine the algorithms and network configuration as necessary to maintain optimal power consumption and network performance.
\end{enumerate}

Through exploring the factors influencing transmission power and employing algorithmic approaches, this research seeks to provide a comprehensive understanding of power optimization in a Thread network and its implications on energy efficiency and network performance in the context of the MOOD-Sense project.


\section{Hardware Analysis}

In this research, several hardware components were employed to implement and optimize the Thread network for power efficiency. Each hardware piece played a crucial role in different stages of the project:

\subsection{\texorpdfstring{\gls{nRF}}{nRF}52840 \texorpdfstring{\acrlong{DK}}{DK}}

The nRF52840 \gls{DK} is a versatile single-board \acrlong{DK} for Bluetooth 5, Bluetooth mesh, Thread, Zigbee, 802.15.4, \gls{ANT}, and 2.4 $GHz$ proprietary applications on the nRF52840 \gls{SoC} \cite{Semiconductor_Nordic_Product_Brief_2018_2.0}. In this research, the nRF52840 \gls{DK} was used to develop and test the Thread network, acting as routers and end devices in the network topology. The \acrlong{DK} enabled the research team to implement and evaluate the network performance and power consumption under different configurations.

\begin{figure}[H]
    \centering
    \includegraphics[width=0.3\textwidth]{images/situational_theoretical_analysis/nRF52840_DK.png}
    \caption{\gls{nRF}52840 \gls{DK} \cite{Semiconductor_Nordic_Product_Brief_2018_2.0}.}
    \label{fig:nRF52840_DK}
\end{figure}

\subsection{nRF52840 Dongle}

The \gls{nRF}52840 Dongle is a small, low-cost \gls{USB} device for the \gls{nRF} Connect for Desktop \gls{PC} tool. It supports Bluetooth 5, Bluetooth mesh, Thread, Zigbee, 802.15.4, \gls{ANT}, and 2.4 $GHz$ proprietary protocols \cite{Semiconductor_Nordic_Dongle_Brief_2018_2.0}. In the research, the Dongle was used to extend the network by adding more nodes, facilitating the evaluation of scalability and network performance in larger network configurations.

\begin{figure}[H]
    \centering
    \includegraphics[width=0.3\textwidth]{images/situational_theoretical_analysis/nRF52840_Dongle.png}
    \caption{\gls{nRF}52840 Dongle \cite{Semiconductor_Nordic_Dongle_Brief_2018_2.0}.}
    \label{fig:nRF52840_dongle}
\end{figure}

\subsection{\texorpdfstring{\acrlong{PPK}}{PPK} II}

The \acrfull{PPK} II is an easy-to-use tool for measuring and optimizing the power consumption of \gls{IoT} devices \cite{Semiconductor_Nordic_PPK_II_2018_1.0}. In this research, the \gls{PPK} II was utilized to measure the power consumption of the \gls{nRF}52840 \gls{DK} devices in various network configurations, enabling the research to assess the energy efficiency of the network and identify areas for improvement.

\begin{figure}[H]
    \centering
    \includegraphics[width=0.3\textwidth]{images/situational_theoretical_analysis/nRF_Power_Profiler_Kit_II.png}
    \caption{\gls{PPK} II \cite{Semiconductor_Nordic_PPK_II_2018_1.0}.}
    \label{fig:nRF_Power_Profiler_Kit_II}
\end{figure}

\subsection{Raspberry Pi 4}

The Raspberry Pi 4 model B is a single-board computer used for various applications, including \gls{IoT} development \cite{alm2019internet}. In this research, the Raspberry Pi 4 served as a border router and provided an interface between the Thread network and external networks. The Raspberry Pi 4 allowed the research to evaluate the overall network performance and data exchange with external systems.

\begin{figure}[H]
    \centering
    \includegraphics[width=0.3\textwidth]{images/situational_theoretical_analysis/Raspberry_Pi_4.jpg}
    \caption{Raspberry Pi 4 model B \cite{alm2019internet}.}
    \label{fig:Raspberry_Pi_4}
\end{figure}

By understanding the roles of each hardware component in the research, it becomes evident how they collectively contributed to the successful implementation and optimization of the Thread network for power efficiency.

\subsection{Constraints and Limitations}

Using these hardware components poses certain constraints and limitations on the research, with some potential consequences:

\begin{enumerate}
    \item \textbf{Limited Scalability}: The number of available \gls{nRF}52840 \gls{DK} and \gls{nRF}52840 Dongle devices may limit the size of the network being optimized, potentially affecting the generalizability of the results. This limitation might make it challenging to extrapolate the findings to larger networks or different device types.
    \item \textbf{Hardware-Specific Performance}: The optimization results might be influenced by the specific hardware used, such as the \gls{nRF}52840 \gls{DK} and \gls{nRF}52840 Dongle, and may not be directly applicable to other devices or platforms. As a consequence, further research and testing may be required to confirm the findings' applicability in different hardware contexts.
    \item \textbf{Measurement Accuracy}: The accuracy of the \gls{PPK} II may impact the precision of the power consumption measurements, potentially affecting the optimization results. This limitation could lead to underestimation or overestimation of energy savings, influencing the overall conclusions regarding the network's energy efficiency.
\end{enumerate}

These hardware-related challenges could influence the research outcomes, making it essential to be aware of the limitations and consider their potential impact on the findings when interpreting the results and applying them to real-world scenarios.

\subsection{Implications for Wireless Network Development}

Taking into account the hardware constraints and limitations, the implications for wireless network development in the context of this research can be examined. The chosen hardware components, such as the \gls{nRF}52840 \gls{DK}, and \gls{nRF}52840 Dongle directly impact the energy efficiency, performance, and scalability of the Thread network. Using these components enabled the investigation and optimization of power consumption and network performance. However, it is essential to acknowledge that hardware limitations might pose challenges when adapting the network to various \gls{IoT} applications or scaling it to larger configurations. By addressing the sub-research question 1, this study highlights the importance of hardware selection and its implications for future wireless network technology development.

\section{Literature Research}\label{sec:literature_research}

Thread network power consumption research has been limited but offers promising results. One study by \textcite{semiconductor_battery_2021} demonstrates that the battery life of a Thread node is heavily dependent on the network configuration. For example, a node with an idle current of 3 $\mu A$ and a transmit current of 17 $mA$ can last up to 10 years in a network with a low data rate of 250 $kbps$ and a small number of packets per day. However, in a network with a high data rate of 1 $Mbps$ and many packets per day, the same node would only last for a few months. A white paper by \textcite{Thread_Low_Power_2018} provides noteworthy results on the power consumption and optimization of Thread networks, showing that the Thread protocol could achieve a standby power consumption of less than 3 $mW$, with typical transmit and receive power consumption ranging between 15 $mW$ and 20 $mW$. The study also demonstrated that devices on a Thread network could achieve up to 10 years of battery life when transmitting once per minute, making Thread a strong candidate for low-power \gls{IoT} applications. Another research effort, conducted by \textcite{azoidou2017battery} analyzed the power consumption of Thread end devices, routers, and coordinators. The study demonstrated that enabling power management features could reduce power consumption by up to 70\% in sleep mode. Additionally, two power optimization techniques, dynamic power management and dynamic voltage and frequency scaling, were evaluated, with the latter having a greater impact, reducing consumption by up to 35\%. The research also emphasized that power consumption is influenced by transmission power level, data rate, and routing topology and suggested that implementing optimization techniques could reduce power consumption by up to 70\%.

In the paper by \textcite{sheth2002implementation} presents a practical implementation of transmit power control for 802.11b wireless networks. They focus on optimizing transmit power to reduce power consumption while maintaining correct reception of packets. The researchers achieved a maximum power savings of 25\%, including idling power, by implementing their adaptive transmit power control algorithm as a user-level application layer process. This work is relevant to power optimization strategies in \gls{IoT} applications, such as Thread networks. \textcite{1576539} investigate the impact of transmission power on the throughput capacity of finite ad hoc wireless networks using a scheduling-based \gls{MAC} protocol in their paper, such as \gls{TDMA}. The authors demonstrate that by properly increasing the nodal transmit power level, the capacity of an ad hoc wireless network can be maximized, regardless of nodal distribution and traffic pattern. The primary finding is that higher transmission power contributes to increased combinatorial diversity, optimizing joint scheduling and routing schemes, which is valuable for the development of efficient \gls{IoT} applications using protocols like Thread.

In the realm of algorithm optimization, the \gls{MCM} is a robust, efficient, flexible, and scalable tool used across various fields, including science, finance, and engineering. Research by \textcite{kroese2014monte} emphasizes \gls{MCM}'s popularity and its applications in areas like industrial engineering, operations research, physical processes, random graphs, finance, biology, medicine, and computer science. The authors highlight \gls{MCM}'s simplicity, strength in randomness, and theoretical justification. \textcite{girgis2014solving} propose a \gls{GA} and \gls{SA} for solving the \gls{WMN} design problem. The study aims to minimize cost and determine the number of used gateways in \gls{WMN} under constraints, with experimental results proving the effectiveness of \gls{GA} and \gls{SA} in minimizing network costs while satisfying quality of service. The authors find that GA outperforms \gls{SA} in small-size networks, while \gls{SA} performs better in large-size networks. On the other hand, \gls{GA} is a heuristic optimization algorithm that handles non-linear, non-convex, and intermittent problems. It is widely applied in various engineering and scientific applications. One study by \textcite{ferentinos2005energy} employs \gls{GA} to optimize \glspl{WSN} for precision agriculture applications. The research determines active sensors, cluster heads, and signal ranges while considering network connectivity, energy conservation, and application requirements. Results indicate that \gls{GA}-generated designs outperform random deployments regarding connectivity and energy consumption. \textcite{norouzi2014genetic} explore the potential of \gls{GA} in optimizing the operational stages of \glspl{WSN}, discussing node placement, network coverage, clustering, data aggregation, and routing. Simulations demonstrate that \gls{GA}-based approaches outperform existing protocols, suggesting that \gls{GA} can optimize \glspl{WSN} in military, medical, and commercial applications.

In summary, although the literature on Thread power optimization is limited, the results from existing studies suggest that the protocol has significant potential for reducing energy consumption in low-power wireless networking applications. For instance, the research by \textcite{sheth2002implementation} demonstrates the effectiveness of optimizing transmit power to reduce power consumption, emphasizing the potential of using algorithmic approaches for power optimization in \gls{IoT} applications such as Thread networks. Furthermore, the research by \textcite{girgis2014solving} shows that \gls{GA} can effectively minimize network costs while satisfying quality of service, highlighting the potential for \gls{GA} optimization in similar wireless network scenarios. Given the success of \gls{MCM} and \gls{GA} in other optimization scenarios, they were chosen to be explored within the context of Thread network power optimization, building upon the limited available literature and attempting to address the gaps in knowledge. This research aims to contribute valuable insights and drive further advancements in the field by applying the proven effectiveness of \gls{MCM} and \gls{GA} in power optimization, aligning with the findings of related studies and enhancing the potential for optimizing power consumption in low-power \gls{IoT} applications.
